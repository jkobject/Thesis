% Fix for URLs and long text going outside page margins
\sloppy % Allow more flexible line breaking
\emergencystretch=1em % Extra stretch for line breaking

\titleformat
{\chapter} % command
[display] % shape
{\bfseries\Huge} % format
{ } % label
{2ex} % sep
{
    %\vspace{1ex}
} % before-code
[ \vspace{0ex}
] % after-code

\raggedbottom % Allow flexible page heights to reduce underfull vbox warnings

\chapter[scPRINT-2: Towards the next-generation of cell foundation models and benchmarks]{\gls{scPRINT}-2: Towards the next-generation of cell foundation models and benchmarks}
\label{article3}

\section{Summary}

Cell biology has been booming with foundation models trained on large
single-cell \gls{RNA-seq} databases, but benchmarks and capabilities remain
unclear. We propose an additive benchmark across a gymnasium of tasks to
discover which features improve performance. From these findings, we
present \gls{scPRINT}-2, a single-cell Foundation Model pre-trained across 350
million cells and 16 organisms. Our contributions in pre-training tasks,
tokenization, and losses made \gls{scPRINT}-2 state-of-the-art in expression
denoising, cell embedding, and cell type prediction. Furthermore, with
our cell-level architecture, \gls{scPRINT}-2 becomes generative, as
demonstrated by our expression imputation and counterfactual reasoning
results. Finally, thanks to our pre-training database, we uncover
generalization to unseen modalities and organisms. These studies,
together with improved abilities in gene embeddings and gene network
inference, place \gls{scPRINT}-2 as a next-generation cell foundation model.

\section{Introduction}\label{introduction}

For the last few years, Single-Cell Foundation Models (\gls{scFM}s), also
known as Virtual Cell models, have provided early approaches to modeling
the cell using single-cell RNA-seq data as their primary
modality\supercite{cuiScGPTBuildingFoundation2024,theodorisTransferLearningEnables2023,yangScBERTLargescalePretrained2022a,rosenUniversalCellEmbeddings2023}. The field has been booming with these
transformer-based machine learning models trained on large databases of
tens of millions of cells. The models themselves contain tens to
hundreds of millions of parameters and are trained on unsupervised (or
semi-supervised) tasks such as predicting masked gene expression or
denoising expression. They can then be used as is to examine their
learned representations or fine-tuned to transfer their knowledge across
a range of everyday tasks in that modality. Many examples have now been
proposed, such as predicting single-cell perturbation responses, patient
drug responses, and disease states; annotating cells; correcting for
batch effects; improving noise levels; imputing unseen gene expression
or modality; generating gene networks; identifying cell niches; and
more\textsuperscript{5,6,6--15}.

While many AI Virtual Cell models and \gls{scFM}s exist, little has been done
regarding their comparison\textsuperscript{16--20}. A crucial question
remains: how to validate the impact of the different proposed methods,
regardless of implementation, datasets, or model size. Indeed,
reproducing results has been challenging for many, and the literature
has yielded discordant conclusions about the performance and
capabilities of these models. Showing they often underperform simpler
approaches on classification, batch correction, and perturbation
prediction\textsuperscript{16,17,21--23}. Much work remains to get to
feature-rich, easy-to-use \gls{scFM}s. Models that allow inference in minutes,
along with well-crafted reproducible benchmarks that demonstrate how
\gls{scFM}s uniquely solve essential problems in single-cell biology.
Open-sourcing not just model weights but their pre-training tasks and
datasets.\textbf{\hfill\break
}

On this front, scPRINT was released as part of a second batch of scFM,
presenting contributions in terms of usability and reproducibility while
also showcasing pre-training strategies, data encoding, and
decoding\supercite{scprint}. scPRINT was trained on 50 million cells
using a multitask pre-training strategy that included expression
denoising, autoencoding, and cell-label prediction. It also presented an
in-depth benchmark that examined the foundation model's zero-shot
performance on these tasks, as well as its internal gene network
representation and fidelity compared to multiple ground truths.

Building on these strengths and moving towards the next generation of
\gls{scFM}s, we here use scPRINT (which will be referred to as scPRINT-1) as
the reference to showcase an extensive additive benchmark of scFM
attributes. We address several key questions about the importance of
diverse architectures, datasets, and training modalities. This additive
benchmark aims to understand the relative importance of these different
features in our task gymnasium, examining the choice of model
architecture and pre-training tasks across 42 different scenarios. In
these scenarios, we propose a breadth of novel components for \gls{scFM}s. In
addition to those 12 distinct contributions, we also examine various
pre-training datasets, compiling a 350-million-cell database---the
largest to date---with over 16 organisms.

As a result of the benchmark, we derive a next-generation scFM,
	extbf{\gls{scPRINT}-2}. \gls{scPRINT}-2 improves upon the previous generation of
models by leveraging our database, the \gls{scPRINT}-2 corpus, and multiple
data augmentation approaches. It uses a set of updated pre-training
tasks and losses, improving its accuracy in challenging and unseen
contexts. Finally, it is equipped with graph-based encoders and the
XPressor architecture, enabling unprecedented expression imputation,
high-quality zero-shot embeddings, and counterfactual reasoning. We dive
into these specific contributions by examining multiple use cases,
highlighting behaviors that are often overlooked or under-assessed in
classical benchmarks.

\gls{scPRINT}-2, its dataloader, pre-training datasets, preprocessing, task
functions, pre-trained weights, as well as the additive benchmark
training traces and all 42 models'{} weights are fully
open-sourced and available under the GPL-v3 License.

\section{\texorpdfstring{\textbf{Results}}{Results}}\label{results}

\subsection{Decoding the impact of a foundation model's architecture
through an additive
benchmark}\label{decoding-the-impact-of-a-foundation-models-architecture-through-an-additive-benchmark}

Many \gls{scFM}s have been developed in single-cell genomics. They have mostly
been studied in isolation, using their own benchmarks. While most of
them maintained relatively similar architectures, the impact of each
design's decisions was never thoroughly assessed. For example, scPRINT-1
uses a denoising reconstruction task similar to scFoundation. Still,
scFoundation uses the mean-squared-error (\textbf{MSE}) for the
reconstruction loss, whereas scPRINT-1 uses the zero-inflated
negative-binomial loss (\textbf{\gls{ZINB}}) (see
\hyperref[zinbmse-loss]{Methods}). scGPT and Geneformer utilize
masking, but scGPT bins expression counts (\textbf{binning}), while
scPRINT-1 does denoising and employs a continuous embedding with a log
transform and a pseudocount of 1 (\textbf{logp1})\supercite{cuiScGPTBuildingFoundation2024,theodorisTransferLearningEnables2023}.
Other models, like cellPLM, instead use a contrastive learning approach,
which encourages embeddings of perturbed and unperturbed cell profiles
to be more similar to each other than those of different cell
profiles. This method is also known as InfoNCE or
Contrastive Cell Embedding (\textbf{CCE}) (see
\hyperref[embedding-contrastive-loss]{Methods})\supercite{oordRepresentationLearningContrastive2019}.

\subsubsection{Additive benchmark}\label{additive-benchmark}

To address the lack of a consistent assessment of these models, we have
designed a benchmark to comprehensively evaluate the various components
of \gls{scFM}s, including pre-training databases, architectures, and training
tasks. This benchmark is based on a gymnasium of tasks similar to those
presented in Kalfon et al.\supercite{scprint} (see Figure \ref{fig:scprint2_overview}; see Table
\ref{tab:additive_benchmark}). The \gls{scFM} gymnasium assesses each model's ability to predict labels,
remove batch effects, denoise, and impute gene expression, as well as
discover known gene-gene relationships at different stages of training.
For embeddings and cell type classification, we use the \gls{scIB} and
accuracy scores over the same ground-truth test datasets as in Kalfon et
al. (see \hyperref[additive-benchmarks-datasets]{Methods}). For
denoising, we evaluate the model's ability to recover the noised
expression profile of cells from a test dataset, as measured by the
improvement in correlation with the ground-truth profile after
denoising. For gene-network inference, we examine the Odds Ratio (OR)
and \gls{AUPRC} scores of the model's ability to recover a ground-truth gene
network from expression data alone (see
\hyperref[gene-network-metrics]{Methods}).

\begin{figure}[H]
    \centering
    \includegraphics[width=0.9\textwidth]{figures/scprint2/image4.png}
    \caption[Presentation of the \gls{scPRINT}-2 model, pre-training dataset, and additive benchmark]{(a) The additive benchmark example table with its gymnasium scores across the \gls{scFM}'s features. (b) Our \gls{scPRINT}-2 corpus pre-training dataset, with 16 organisms across 300+ tissues. \gls{UMAP} of 15 million cells from the corpus integrated using \gls{scPRINT}-2. Colors represent species. (c) The \gls{scPRINT}-2 model, its input data, and its different outputs. Source data are provided as a Source Data file.}
    \label{fig:scprint2_overview}
\end{figure}

The base model, on which the additive benchmark is performed (see Figure
\ref{fig:scprint2_overview}, Table \ref{tab:additive_benchmark}, and \hyperref[base-model-and-training]{Methods}), is
trained on the CxG database, comprising 500 carefully annotated human
and mouse datasets. Its training lasts for a maximum of 20 epochs, each
of 20,000 steps, with a minibatch size of 64. We encode the gene
expression using the scPRINT-1 approach and decode it with the MSE
method. The base model's pre-training task uses a 30\% gene expression
mask. We pre-train the models 6 times across multiple seeds to generate
error bounds. Using Flash-Attention-3, the 20M parameters model trains
on 1 H100 \gls{GPU} for 2 days.

While we will not delve into the details of each feature assessed (see
\hyperref[additive-benchmark-1]{Methods}), our benchmark broadly
highlights several key points.

Regarding the tasks, we have confirmed what Kalfon et al. and De Waele
et al. previously showed: that denoising is superior to masking as a
pre-training task for single-cell data in classification and embedding
tasks\supercite{scprint,deWaeleSystematicAssessment2025}. Similarly, un-normalized expression is
better than normalizing it at the input. Classification also serves as a
good supplement to the pre-training task, as without it, we observe a
slight decrease in performance (see Table 1).

We also present, as part of our study, the \textbf{\gls{scPRINT}-2 corpus},
which comprises more than 350 million single cells (see Figure \ref{fig:scprint2_overview}). This
is the largest dataset ever assembled, consisting of data from the Chan
Zuckerberg Institute's Cellxgene (\textbf{CxG}), the \textbf{Tahoe}-100M
dataset, and the scBasecount database, which contains 20,000 reprocessed
datasets from the Gene Expression Omnibus\supercite{programCZCELLxGENEDiscover2023,youngblutScBaseCount2025,zhangTahoe100M2025}. The
cells themselves are derived from 16 different eukaryotic organisms,
spanning more than 1 billion years of evolution. The dataset comprises
approximately 400,000 distinct genes, 4,764 different labels, and around
140,000 cell groups, totaling 25 TB of unique data\supercite{jeremiekalfonTrainingFoundationModels}.
Our database contains nine main classes: \emph{cell type, disease, age,
tissue of origin, assay, ethnicity, sex, cell culture,} and
\emph{organism}.

Thanks to this database, we demonstrated the growing importance of data
selection in pre-training \gls{scFM}s. Indeed, when using the Tahoe-100M
database solely for pre-training, the model's overall performance
plummets, as the sequencing depth and diversity are low despite the
large number of cells.

However, including this lower-diversity dataset with the high-diversity
CxG database and carefully considering the cell-state imbalances results
in only a noticeable decrease in denoising performance. Interestingly,
using all available datasets did not change performance across our
benchmarks. Reducing the training database to a random subset of only
\textbf{200 human datasets only,} led to a minimal decrease in denoising
and cell type prediction. This shows again that the benchmark fails to
highlight abilities on more diverse cell types and
organisms\textsuperscript{30,31}. But it also indicates diminishing
returns in adding more datasets---diversity in cell states and organisms
being much more important than cell count.

We thus preprocessed each dataset by removing all duplicates, filtering
for low-quality cells, aligning metadata to the CxG ontologies, and
computing cell-cell similarity profiles and clusters. It allowed us to
introduce multiple data augmentation techniques, such as varying the
input context length (\textbf{var. context}) during training and
randomly creating \textbf{meta-cells}, which are averages of similar
cell expression profiles across K-nearest neighbors (K-NN) (see
\hyperref[a-multi-cell-denoising-auto-encoder-task-unlocks-new-modalities-and-performances]{Results
section 3}). Interestingly, we observe that both methods tend to
improve the model's performance in most metrics, even though these
models do not examine more cells overall. This highlights the importance
of effective data augmentation techniques for scFM
pre-training\textsuperscript{32}.

\begin{figure}[H]
    \centering
    \includegraphics[width=0.9\textwidth]{figures/scprint2/image6.png}
    \caption[Full results of the additive benchmark]{Table representing the results of the additive benchmark on 42 models, over multiple metrics: batch correction and cell embedding quality, denoising quality, cell type prediction, and gene network inference. Additional information on the different components is available in the methods section. Bold elements are the features that are part of the \gls{scPRINT}-2 foundation model.}
    \label{tab:additive_benchmark}
\end{figure}

Regarding architecture, we recomputed results from the XPressor
manuscript\textsuperscript{33}, which showed that this architecture
improves the embedding quality of \gls{scFM}s (see
\hyperref[an-efficient-hierarchical-attention-architecture-makes-scprint-2-generative]{Results
section 4}; see the full table in Supplementary Table~\ref{table-s1-detailed-version-of-the-additive-benchmark}). We also demonstrate
that using ESM-based gene ID tokens leads to much better performance
than learning gene tokens from scratch\textsuperscript{34}. Providing
each gene's genomic location as additional input information
significantly improves model convergence. However, we also noticed that
when they do converge, models without gene location information can
perform well. We have noticed that model size correlates with higher
scores, at least for gene network inference and cell-type prediction.
Using a Graph Neural Network (\textbf{GNN}) encoder shows significant
improvements, with only a slight decrease in the cell-type prediction
task (see Results Section 3; see \hyperref[methods]{Methods}).
Additionally, our sub-quadratic attention mechanism, Criss-cross
attention, also shows substantial benefits with no reduction in
performance (see Results section 4; see \hyperref[methods]{Methods}).

Moreover, \gls{MSE}, on average, outperforms \gls{ZINB} as a loss function while
decreasing the model's expressivity (see
\hyperref[base-model-and-training]{Methods}). A good proposed
middle ground is the \gls{ZINB}+MSE loss (see
\hyperref[a-multi-cell-denoising-auto-encoder-task-unlocks-new-modalities-and-performances]{Results
Section 3}; see \hyperref[zinbmse-loss]{Methods}).

Some unexpected results showed that omitting the decoder part of
scPRINT-1 led to stronger performance; however, this comes at the cost
of generative abilities and decreased cell-embedding fidelity. Indeed,
despite its importance for understanding \gls{scFM}s' behavior and feature
importance, we have noted that our benchmark does not yet capture the
full breadth of abilities that \gls{scFM}s do or should have. For example,
both \gls{scIB} and classification scores are very dependent on the
dataset' s quality and its labels. Scores presented here
show only a facet of the model's ability. We might be interested in the
model's performance up-to-convergence instead of stopping them at 20
epochs or looking at unseen species, or assays at training. This is a
first attempt to benchmark \gls{scFM}s, but more extensive efforts will be
needed.

\subsubsection{\gls{scPRINT}-2}\label{scprint-2}

Overall, we have examined the performance improvements driven by our 12
distinct contributions across 42 training runs. Based on these results
and our own considerations, we have elected a set of features to create
\gls{scPRINT}-2, a next-generation cell foundation model (see Supplementary Figure~\ref{fig-s1-illustration-of-the-full-scprint-2s-architecture-input-and-output};
see \hyperref[scprint-2-1]{Methods}). We highlight its architecture
in Figure \ref{fig:scprint2_overview}; \gls{scPRINT}-2 is currently available in a small version with
only 20M active parameters. Its encoder-compressor-decoder architecture
produces cell- and gene-level outputs at multiple levels, working on one
or more cells at a time.

Furthermore, to aid in the exploration of this largest-ever
cross-organism single-cell dataset, we release all of the 350 million
cells in the \gls{scPRINT}-2 corpus, aligned into an atlas by \gls{scPRINT}-2, of
which 1\% are directly accessible through an interactive visualization
(see Figure \ref{fig:scprint2_overview}, see \hyperref[data-availability]{Data
availability}) along with \gls{scPRINT}-2 cell label predictions for all
classes. This should enable never-before analysis and exploration of
single-cell RNA-seq data.

But the additive benchmark leaves some questions unanswered about the
effect of combining these features up-to-convergence and the models'
abilities on unseen modalities, tasks, and species. In the following
sections, we will focus on 1. looking at more diverse and truthful
datasets in size, quality, and source domains; 2. using more scores and
ground truth validations; 3. defining tasks that better reflect the
possibilities and real-life use of these models.

\subsection{A diverse dataset of 350 million cells pushes generalization
to unseen
organisms}\label{a-diverse-dataset-of-350-million-cells-pushes-generalization-to-unseen-organisms}

One of the most critical features of foundation models (FMs) is the
breadth of their training dataset. From vision to language, AI
advancement has been driven by training models on ever-larger
datasets\textsuperscript{35--39}. Nowadays, most \gls{scFM}s are trained on 20
to 50 million cells, except the recently released Geneformer-v2 and
STATE-SE models, which have been trained on roughly 300 million
cells\textsuperscript{40,41}.

\subsubsection{scPRINT-2 pre-training
corpus}\label{scprint-2-pre-training-corpus}

In conjunction with our model's architecture, the \gls{scPRINT}-2 corpus and
its 16 organisms enable generalization to organisms unseen during
training. This broader cell type diversity, however, comes with
additional challenges: annotation quality has decreased due to missing
annotations in scBasecount. Additionally, the skew toward low sequencing
depth and highly similar cells has increased with the inclusion of
spatial transcriptomics datasets and less curated databases such as
Tahoe-100M and Arc's scBasecount (see
\hyperref[depth-weighted-sampling]{Methods}).

Fortunately, a key feature of our dataloader,
scDataLoader\textsuperscript{3}, is its ability to perform weighted
random sampling, thereby mitigating the heavy dataset imbalances that
currently exist across diverse cell types, sequencing methodologies, and
different organisms assessed. We thus present methods to successfully
train \gls{scPRINT}-2 on this large dataset. The first, called
cluster-weighted sampling, allows datasets with unclear annotations to
benefit from weighted random sampling by defining clusters of high
expression similarity (see
\hyperref[cluster-weighted-sampling]{Methods}). This lets us define
cell states without requiring any label information and perform sampling
that is aware of the different cell states, regardless of the size of
each cluster. We address the second issue of uneven cell quality by also
skewing sampling toward cells with more non-zero genes (nnz). Both
methods were enabled on such a vast database thanks to essential updates
to scDataLoader. This re-weighting is performed jointly with weights on
cell type, disease, organism, and sequencer labels, thereby addressing
the size/diversity issues that plague these larger cell
databases\textsuperscript{42}.

Interestingly, the number of training steps required to achieve
convergence increased only 2-fold, indicating that, as in scPRINT-1, the
model did not sample as many cells as actually exist in the pre-training
dataset before reaching convergence. However, with data augmentation and
nearest-neighbor sampling, the model still encountered roughly 2 billion
distinct input cell profiles during pre-training, corresponding to 2000
cell profiles per step.

After implementing this feature and training \gls{scPRINT}-2, its cell-type
classification performance on the validation dataset was 76\%. For its
other predicted labels, its performance was 59\% (disease), 96\%
(ethnicity), 96\% (assay), 94\% (age), 100\% (cell culture), 100\%
(organism), 93\% (sex), and 70\% (tissue of origin).

\begin{figure}[H]
    \centering
    \includegraphics[width=0.9\textwidth]{figures/scprint2/image2.png}
    \caption[Presentation of the updated classifier and results on classification tasks]{(a) Open Problems benchmark results and comparison of scPRINT-1 and zero-shot and fine-tuned \gls{scPRINT}-2. (b) Illustration of our updated hierarchical classifier loss. (c) Unseen organisms cell type classification for cat and tiger datasets, across two experts and \gls{scPRINT}-2 zero-shot, after label smoothing, after cluster aggregation, and after fine-tuning. (d) Change in classification accuracy as the number of genes in context increases for high-quality single-cell datasets. The star represents the model's score when label smoothing is used.}
    \label{fig:scprint2_classifier}
\end{figure}

On the live benchmark Open Problem from November 2025, it achieved an
average zero-shot performance of 75\%, putting it above scPRINT-1 (47\%)
and other zero-shot FMs (40-60\%), even above Liger, a supervised
technique\textsuperscript{43} (see Figure \ref{fig:scprint2_classifier}; see
\hyperref[open-problem-benchmarks]{Methods}). But \gls{scPRINT}-2 was the
only \gls{scFM} with UCE that could run on all datasets\textsuperscript{44}.
Against the two human datasets on which Scimilarity-KNN could be run, it
performed slightly better than \gls{scPRINT}-2. This is most likely due to the
smaller capacity of \gls{scPRINT}-2 (20M parameters) compared to scimilarity
(100M parameters), as we also observed in our additive benchmark (see
Table \ref{tab:additive_benchmark}). Another likely reason is that the model likely saw those
datasets more often during pre-training, since it is trained only on
CxG's human datasets.

We then performed fine-tuning using our XPressor-based
Parameter-Efficient Fine-tuning (XPEFT), in which we fine-tune only the
XPressor layers of \gls{scPRINT}-2 (see
\hyperref[fine-tuning-task]{Methods}). In this context, we show
that \gls{scPRINT}-2 fine-tuned outperforms every existing supervised and
unsupervised method on the Open-problem (see Part 4; see
\hyperref[xpressor-model]{Methods})\textsuperscript{45}. We
observed similar trends in the macro-F1 scores (see Supplementary Figure~\ref{fig-s2-barplot-of-the-f1-macro-scores-on-the-label-projection-task-of-the-open-problem-benchmark}). Of
note, neither scGPT nor Geneformer are currently tested in their
fine-tuned version on the platform.

These performances are enabled in part by our update to scPRINT-1's
hierarchical classification loss (see Figure \ref{fig:scprint2_classifier}). The scPRINT-1
classifier generates predictions for all possible labels in a
hierarchical ontology, while producing logits only for the leaf labels.
To predict the other labels, it only has to aggregate their leaf logits.
In \gls{scPRINT}-2, we improve on this loss by using the entire ontological
graph, meaning that, e.g., given a ground truth of \emph{olfactory
neuron,} we will penalize a prediction of \emph{inhibitory neuron} less
overall than a non-neuron label, like \emph{fibroblast}. In conjunction
with our weighted sampler, this allows the model to learn rich gradients
from a low volume of data.

\subsubsection{\gls{scPRINT}-2 generalizes to unseen classification
tasks}\label{scprint-2-generalizes-to-unseen-classification-tasks}

We have, however, noticed that classification performance does not
generalize sufficiently to correctly recover the exact phylogenetic
relationships within organisms or, similarly, within ethnicities (see
Supplementary Figures~\ref{fig-s3-heatmap-of-ethnicity-prediction-relationship-across-samples}, \ref{fig-s4-heatmap-of-organism-prediction-relationship-across-samples}, \ref{fig-s5-heatmap-of-organism-prediction-relationship-using-organism-embedding-similarity-across-samples}). This could be biased heavily by tissue
representation in rare ethnicities and organisms. However, some
relationships were found, such as \emph{Singaporean Indian/Singaporean
Chinese, Korean/Japanese/Chinese, American/Latin American,} or
\emph{Macaque/Marmoset/Chimpanzee, Drosophila/C. elegans, Human/Mouse,
Pig/Cow}, suggesting that with greater diversity and representation,
\gls{scFM}s might learn this relationship classification of gene expression on
their own.

We show that this does not prevent \gls{scPRINT}-2 from generalizing to unseen
organisms. Using a randomly selected tomato plant dataset and its
corresponding ESM3 gene embeddings, unseen at training time, \gls{scPRINT}-2
generates an organism label prediction for the two plant organisms it
knows about 67\% of the time. This is despite the very low prevalence of
these organisms in the pre-training dataset (see Figure \ref{fig:scprint2_overview}). For a horse
dataset, \gls{scPRINT}-2 predicted mammalian organisms 72\% of the time.

Unfortunately, these datasets lacked cell-type annotations. Using
well-annotated datasets from Zhong et al.\textsuperscript{46} of cat and
tiger lung tissues, organisms not seen at training time, we generate
cell type predictions using \gls{scPRINT}-2 and achieved a prediction accuracy
of 42.2\% across the 500 potential cell type leaf labels \gls{scPRINT}-2 knows
about. While this score may seem low compared to supervised approaches,
it is worth noting that labels from a secondary source were available in
the datasets. Comparing them to the initial ground truth, we found only
a 55.3\% agreement between the two. Furthermore, we noticed that for
some cells, annotations were quite different, such as: \emph{fibroblast}
being labelled as \emph{ciliated cell}, \emph{macrophage} as
\emph{neuroendocrine cell, and} \emph{ionocyte} as \emph{secretory
cell}.

Given the low correspondence between the two expert annotations, we
wanted to determine which was correct between \gls{scPRINT}-2 zero-shot or the
expert ground-truth labels. We conducted a differential expression
analysis between cells labeled as \emph{type 2 pneumocyte} by \gls{scPRINT}-2
(zero-shot) but as \emph{macrophage} by the ground truth (see Supplementary
Figure~\ref{fig-s6-differential-expression-plots-of-the-disagreeing-cells-between-scprint-2-and-ground-truth}). We saw that the most highly differentially expressed genes
were \emph{MAGI1}, \emph{NPNT}, \emph{TEAD1}, and \emph{LMO7}, which are
involved in cell-cell junctions, epithelial cells, alveolar cells, and
lung tissues. Moreover, the first differentially expressed gene was
\emph{SFTPC}, a known ``type 2 pneumocyte'' marker. This means that,
even in this challenging unseen-organism dataset, \gls{scPRINT}-2 seems to
legitimately correct expert annotations. This showcases strong
generalization to unseen organisms.

To further improve \gls{scPRINT}-2's accuracy, we use a method first presented
in Hu et al. to aggregate predictions based on \textbf{nearest neighbor
smoothing} of the model's class logits (see
\hyperref[logits-refinement-laplacian-smoothing]{Methods})\textsuperscript{47,48}.
This approach increased accuracy in most of our use cases but yielded a
small 1.3\% improvement here. We also provide tools to perform
\textbf{top-K predictions} and \textbf{confidence-based selection}. This
means that \gls{scPRINT}-2 can list multiple putative labels for each cell.
When multiple labels have high logits, it can output their shared
parental label for that cell instead. When labels disagree, or the
logits are low, \gls{scPRINT}-2 can output an ``unknown'' label instead. Using
both approaches together, we get an additional 3\% improvement in
accuracy, with 10\% of the cells now listed as ``unknown''.

Additionally, the low accuracy is also related to \gls{scPRINT}-2 predictions
being cell-specific, whereas most ground truth labels are
cluster-specific. We propose a \textbf{cluster-based logits averaging,}
which can be viewed as an extreme case of smoothing (see
\hyperref[cluster-aggregation]{Methods}). With this tool, \gls{scPRINT}-2
performance increased by 12\% (see Figure \ref{fig:scprint2_classifier}). Beyond improved accuracy,
these inference-time contributions significantly enhance the usefulness
of scFM-based cell annotation for biologists.

Finally, we also demonstrate that with our XPEFT method (presented
further in
\hyperref[an-efficient-hierarchical-attention-architecture-makes-scprint-2-generative]{Results
section 4}), \gls{scPRINT}-2 can improve its predictions to 95\% accuracy in
the test subset, while preserving some fine-grained cell-type
distinctions not present in the training data (see Figure \ref{fig:scprint2_classifier}).

We then assessed \gls{scPRINT}-2' s performance as we increased
the number of genes in context. We used a Smart-seq-v4 dataset from
Jorstad et al., averaging around 6000 nnz genes per cell (see Supplementary
Figure~\ref{fig-s7-umap-of-the-smart-seq-dataset-used-in-the-varying-context-classification-task})\textsuperscript{49}. As shown in Figure \ref{fig:scprint2_classifier}, we observed an
overall increase in prediction accuracy across all labels as we
increased the context from 200 to 8000 genes, even though \gls{scPRINT}-2 was
pre-trained on only 3200 genes, demonstrating generalization to larger
input contexts. Interestingly, classes such as sex and ethnicity reached
much better predictive accuracy as we increased the number of genes.
When using only the most expressed genes in context, we observed that
cell types, which are often defined by highly expressed canonical genes,
remained relatively high, even with only 200 genes in context (see Supplementary
Figure~\ref{fig-s8-line-plot-of-the-classification-across-varying-context-length-using-the-most-expressed-genes}).

Training \gls{scFM}s on large dataset sizes does not necessarily improve the
model's performance. It is the breadth of cell types, conditions,
organisms, and cell quality that produces real generalization abilities.
We showcased it here, with scPRINT2 able to label unseen organisms,
improving its predictions across various context lengths and rare
modalities. We also showed \gls{scPRINT}-2 reaching state-of-the-art
classification accuracy with our fine-tuning.

We will now see how some of our contributions in training loss and data
augmentation can similarly improve performance in denoising and
imputation in unseen modalities.

\subsection{A multi-cell denoising auto-encoder task unlocks new
modalities and
performances}\label{a-multi-cell-denoising-auto-encoder-task-unlocks-new-modalities-and-performances}

Not all single-cell datasets are at the sequencing depth and quality of
Smart-seq-v4. On average, single-cell data has very low depth,
preventing \gls{scFM}s from learning features that may only be seen in
higher-quality cellular profiles.

\subsubsection{Meta-cells and graph neural network
encoder}\label{meta-cells-and-graph-neural-network-encoder}

In addition to biasing sampling toward cells with more non-zero genes
(nnz), \gls{scPRINT}-2's dataloader now uses neighborhood information, whether
defined in expression space or via spatial transcriptomics (see Figure
\ref{fig:scprint2_denoising}; see \hyperref[multi-cell-sampling]{Methods}). This allows users
to create models that take into account nearest neighbor cells during
pre-training. This can be done, for example, by creating
\textbf{meta-cells}. Meta-cells average the expression over the cell and
its neighbors to artificially create a higher-depth cell with less
dropout. We demonstrate that this approach achieves improved results
across multiple model metrics, but not in denoising (see Table 1). While
17\% of cells in the dataset have more than 2600 non-zero values, 11\%
had at least 3200. With nnz-weighted sampling, we reach 33\%. By adding
metacells, half of our input expression profiles now have more than 3200
nnz elements---allowing us to extend \gls{scPRINT}-2's context to 3200 genes.

However, one can go beyond meta-cells and, instead of averaging, use a
graph neural network (\textbf{GNN}) (see Figure \ref{fig:scprint2_denoising}; see
\hyperref[gnn-expression-encoder]{Methods})\textsuperscript{50,51}.
In this case, the set of neighbors' expressions is encoded in the input
token of the transformer. We show that this improves the model's
denoising ability. However, we also noticed a decrease in cell embedding
and classification (see Table 1). Further experiments showed that this
was mitigated with longer training time. As in the variable context
case, we variably select 0 to 6 neighbors per minibatch, so the model
learns to use a variable number of cell neighbors (see Supplementary Figure~\ref{fig-s9-illustration-of-the-multiple-perturbations-applied-to-expression-data-in-scprint-2},
see \hyperref[multi-cell-sampling]{Methods} for details on the
choice of neighbors).

\begin{figure}[H]
    \centering
    \includegraphics[width=0.9\textwidth]{figures/scprint2/image5.png}
    \caption[Presentation of the expression encoder and decoders and performance on denoising and imputation tasks]{(a) Overview of \gls{scPRINT}-2's multi-cell expression encoder and (b) \gls{scPRINT}-2's expression decoder loss. Circles represent scalar values, orange blocks represent vectors. (c) Benchmark of \gls{scPRINT}-2 on expression denoising over nine datasets of varying quality, compared to MAGIC and scPRINT-1. (d) \gls{UMAP} of the Xenium's patches of cells' expression pre/post denoising with \gls{scPRINT}-2. (e) Expression denoising of IRF8 with \gls{scPRINT}-2 over a sub-patch of the Xenium melanoma dataset with cell contour overlaid. (f) Overview of the patch selection in the Xenium dataset, and of the mapping and pseudo-imputation with Tangram using a matched melanoma 10x v3 \gls{scRNA-seq} dataset. (g) Correlation-based denoising \& imputation scores of \gls{scPRINT}-2 and denoising of MAGIC on the matched dataset. (h) \gls{scPRINT}-2 cell type prediction over the Xenium melanoma patch. (i) Expression-based clusters and \gls{scPRINT}-2 disease prediction of cells from the Xenium melanoma patch analyzed. Source data are provided as a Source Data file.}
    \label{fig:scprint2_denoising}
\end{figure}

Pushing our analysis further, we realized that a mix of both scores,
which we call \textbf{\gls{ZINB}+MSE} (see Figure 3B; see
\hyperref[zinbmse-loss]{Methods}), yields a better denoising score
while retaining the ability to model zero inflation and uncertainty (see
Table \ref{tab:additive_benchmark}). Together, these updates have already made \gls{scPRINT}-2 better
than scPRINT-1 and even better than MAGIC on our denoising benchmarks
(see Table \ref{tab:additive_benchmark}, see Figure \ref{fig:scprint2_denoising})\textsuperscript{52}. While these results
are already state-of-the-art, we wanted to explore the effects of
denoising and how to assess our model in unseen contexts.

Looking at denoising scores across technologies, we notice that
scPRINT-1 tends to perform much better on datasets with higher nnz
genes, i.e., higher-quality datasets (see Figure \ref{fig:scprint2_denoising}, see
\hyperref[denoising-task]{Methods}). However, within each dataset,
scPRINT-1 struggles more with low-depth cells than MAGIC \& \gls{scPRINT}-2,
which is more consistent overall. We explain this paradox by the fact
that, beyond nnz genes, the high-quality dataset often exhibits lower
biases in the distribution of nnz genes per cell (see Supplementary Figure~\ref{fig-s10-distplot-of-the-non-zero-count-distribution-across-cells-from-the-three-dataset-qualities-used}).
This also explains why MAGIC and \gls{scPRINT}-2 perform better than scPRINT-1
in these biased datasets. Indeed, they can look at the neighbor's
expression and model the expression biases this way. This usage explains
the significant improvements in the low- and mid-quality datasets,
making \gls{scPRINT}-2 state-of-the-art across all tested contexts and
modalities using its estimate of zero-inflation.

\subsubsection{\gls{scPRINT}-2 generalizes to unseen denoising
tasks}\label{scprint-2-generalizes-to-unseen-denoising-tasks}

Additionally, we decided to look at performance on a Xenium dataset, a
modality completely absent from \gls{scPRINT}-2's training (see
\hyperref[xenium-analysis]{Methods})\textsuperscript{53}. We
elected to use a large, recent skin melanoma dataset with a 5000-gene
panel, reaching the upper limit of what is doable with current
technology.

A first proof of \gls{scPRINT}-2's denoising is the \gls{scIB} biological
truthfulness of the Xenium dataset, which improves over the raw
expression embedding when using its embeddings (see Figure \ref{fig:scprint2_denoising}; see
Supplementary Figure~\ref{fig-s11-umap-over-scprint-2-and-pca-embeddings-of-the-xenium-dataset}; see Supplementary Table~\ref{table-s3-detailed-scib-biological-conservation-scores-on-the-xenium-dataset}). To further assess how well
\gls{scPRINT}-2 can denoise this unseen data modality, we leverage the optimal
transport-based method Tangram\supercite{biancalaniDeepLearningAlignment2021}. We used Tangram to
map each Xenium cell to another cell in a non-spatial 10X v3 dataset of
similar skin melanoma\supercite{zhangSinglecellAnalysisReveals2022} (see Figure 3F). Here, the
mapping quality is low due to many differences between the two
technologies, e.g., number of cells, number of genes per cell, or biases
in cell and gene types (see Supplementary Figure~\ref{fig-s12-tangram-mapping-quality-plots}). Still, using the 10X v3
dataset as ground truth, we can see that MAGIC and \gls{scPRINT}-2 recreate an
expression profile that correlates more than 30\% better with the 10X
dataset than does Xenium (see Figure \ref{fig:scprint2_denoising}). There, MAGIC creates
expression profiles closer to the 10X ones, while \gls{scPRINT}-2 remains
closer to the initial Xenium profiles, and both \gls{scPRINT}-2 and MAGIC tend
to agree more with each other than with anything else (see Figure \ref{fig:scprint2_denoising}).

Overall, this suggests that using a tool like \gls{scPRINT}-2 might be a
better alternative for denoising and imputing expression from Xenium
than using a secondary non-spatial 10X dataset and aligning it with
Tangram.

At the same time, MAGIC can only perform denoising and cannot impute
expression for unseen genes. We thus use \gls{scPRINT}-2 to impute a random
subset of 5000 genes present only in the 10X v3 dataset. Interestingly,
we noticed that feeding all 5000 (expressed in Xenium) + 5000
(unexpressed in Xenium) genes in context did not lead to good
imputation. However, using \gls{scPRINT}-2's generative architecture, we
directly decoded the 5000 10X-only genes from the \gls{scPRINT}-2's cell
tokens generated on the 5000 Xenium genes (see Supplementary Figure~\ref{fig-s13-illustration-of-scprint-2s-generative-imputation-mechanism}). We
show that this imputation scores as high as the denoised Xenium genes
(see Figure \ref{fig:scprint2_denoising}).

Finally, we also wanted to examine \gls{scPRINT}-2's cell-label predictions on
this unseen modality. While we did not have access to ground-truth
labels in this dataset, we could already spot-check the validity of the
predictions. Indeed, many cell types were labeled as \emph{basal} or
\emph{epidermis}, with numerous immune cell labels in the cancer-induced
lesion in the tissue (see Figure \ref{fig:scprint2_denoising}). It likely reflects the biases of the pre-training dataset,
where disease labels are often applied at the dataset level rather than
the cell level, making \gls{scPRINT}-2's disease predictions sometimes
imprecise. Thankfully, many cells had the cell-type label
`\emph{malignant cell}'. These cells were distributed throughout the
tissue and showed a strong signal for the five key literature melanoma
genes (\emph{BCL2, IGF1, EGFR, FGFR2, SOX10}) (see Supplementary Figures~\ref{fig-s15-violin-plot-comparison-of-the-genes-expression-between-predicted-malignant-vs-the-rest} and~\ref{fig-s16-differential-expression-plot-of-cancer-disease-labelled-vs-rest-in-the-xenium-dataset}).

Overall, we have seen how \gls{scPRINT}-2 can be used on challenging
modalities to augment a given dataset with cell label predictions,
expression denoising, and gene imputation. Showing yet again another
axis of generalization. We will now focus on how some of our contributions in training loss and data
augmentation can similarly improve performance in denoising and
imputation in unseen modalities.

\subsection{An efficient, hierarchical attention architecture makes
scPRINT-2
generative}\label{an-efficient-hierarchical-attention-architecture-makes-scprint-2-generative}

\subsubsection{Efficient attention architectures and compression
methods}\label{efficient-attention-architectures-and-compression-methods}

Implementing transformer models on new modalities is a potent way to
rethink some of their mechanisms. A common issue with transformer models
is their memory and compute requirements, which grow quadratically with
their context length (e.g., the number of genes in their input). This is
even more pronounced in bidirectional transformers like most \gls{scFM}s. With
the introduction of scPRINT-1, we presented a model that could train in
3 days on a regular A40 \gls{GPU} and on 50M cells, an order of magnitude
faster than most similar \gls{scFM}s. A first contribution to the \gls{scPRINT}-2
architecture is the addition of state-of-the-art approaches to reduce
the memory footprint and increase training speed. We modified the
attention mechanism in multiple ways, using grouped-query attention
(GQA) to reduce memory usage. We benchmarked additional attention
mechanisms alongside Flash-Attention-3 to assess their performance and
their speed.

\begin{figure}[H]
    \centering
    \includegraphics[width=0.9\textwidth]{figures/scprint2/image1.png}
    \caption[Presentation of the XPressor architecture and performance on cell embedding tasks]{(a) Presentation of the XPressor with \gls{VAE}-based compression. (b) Schematic representation of going from expression to classification with \gls{scPRINT}-2, XPressor, and \gls{VAE}-based compression. (c) Open-Problem scores for \gls{scPRINT}-2 across all methods. (d) \gls{UMAP}s of, respectively, \gls{PCA} embeddings, \gls{scPRINT}-2 zero-shot cell-type embeddings, and \gls{scPRINT}-2 fine-tuned cell-type embedding colors by known cell types and batches, with \gls{scIB} total scores. (e) Schematic representation of the counterfactual generation using \gls{scPRINT}-2's embedding and replacing them for the organism class from mouse to human. (f) Illustration of the decrease in distance between initially unrelated datasets from applying this counterfactual approach. (g) Differentially expressed genes post vs pre mouse ``humanization'' with \gls{scPRINT}-2. (h) Over-representation plot of the top positively differentially expressed genes in both human-like mouse and real human vs. mouse; the red line indicates random chance. Source data are provided as a Source Data file.}
    \label{fig:scprint2_xpressor}
\end{figure}

A first one is flash-\textbf{hyper attention}, which computes specific
attention only on sets of keys and queries known to be similar via
locality-sensitive hashing and clustering\textsuperscript{56}. A second
one is flash-\textbf{softpick} \textbf{attention}, a rectified softmax
that decreases hyperactivation of specific tokens, often called
attention sinks\textsuperscript{57}. We also present our own
sub-quadratic attention mechanism: \textbf{criss-cross attention} (see \hyperref[methods]{Methods}),
inspired by advanced concepts such as the Recurrent Interface Network
(RIN) and the Induced Set Attention Block (ISAB)\textsuperscript{58,59}.
It compresses attention by sketching it in context, using a doubly
cross-attention mechanism with a set of latent tokens that get updated
across layers (see Supplementary Figure~\ref{fig-s17-illustration-of-criss-cross-attention}). We show that only criss-cross
attention dramatically improved the model' s speed while
retaining all its abilities (see Table 1). However, it is not yet
compatible to retrieve gene networks from; for this reason, our
\gls{scPRINT}-2 architecture, for now, uses flash-attention-3 and XPressor.

On another direction, while single-cell analysis has leveraged \gls{VAE}s for
years to generate meaningful compressed representations of cells,
transformers inherently lack this ability\textsuperscript{60--62}. We
use the \textbf{XPressor} architecture presented in Kalfon et
al.\textsuperscript{33}, which compresses output gene embeddings into a
set of cell embeddings and decompresses them back into their original
gene embeddings (see Figure \ref{fig:scprint2_overview}, Figure \ref{fig:scprint2_xpressor}, and
\hyperref[xpressor-model]{Methods}). This innovative architecture
draws on ideas that have existed in the transformer literature for
several years\textsuperscript{59,63--66}. We show in our ablation study
that using XPressor results in a slightly better cell representation
overall, but does not meet the statistical threshold. This difference
might be explained by the limit in the number of epochs and the model's
smaller size compared to Kalfon et al. (see Table \ref{tab:additive_benchmark}, see Supplementary Table~\ref{table-s1-detailed-version-of-the-additive-benchmark}). We include an extension to this approach, in which one appends \gls{VAE}s
to each output embedding of XPressor to regularise the different cell
embeddings generated by the model (see Figure \ref{fig:scprint2_xpressor}). This addition allows
us to choose a specific dimension for each cell embedding that is lower
than that of XPressor. A second constraint is defined by applying the
Kullback-Leibler divergence (KL) loss (see Figure \ref{fig:scprint2_overview}, see
\hyperref[vae-based-compressor-model]{Methods}). This creates an
information bottleneck for the different cell embeddings, pushing the
model to select only the minimum amount of relevant information to
represent the label. While our ablation study does not show improvement
in cell embeddings with this approach, this is likely because each
method was trained for only 20 epochs. Indeed, the VAE-infused model is
taking longer to learn to classify cells. However, the batch correction
score improved significantly, indicating that the different cell tokens
mainly contained information about the class they encoded (see Supplementary
Materials). Now that we have highly compressed cell-level embeddings
(i.e., tokens), we can apply a \textbf{dissimilarity loss} between each
for a given cell. This actively pushes them to be as different as
possible (see Supplementary Figure~\ref{fig-s18-illustration-of-the-similarity-and-dissimilarity-based-contrastive-losses-used-in-scprint-2}; see
\hyperref[embedding-independence-loss]{Methods}). We demonstrate
that this tends to slightly improve the model' s output
embedding in our ablation study (see Table 1).

These architectural changes make scPRINT-2 much more efficient at
compression and zero-shot batch-correction. Indeed, on the open
problem's benchmark, we observe an overall improvement over scPRINT-1,
again becoming the state-of-the-art zero-shot method on the platform
(see Figure \ref{fig:scprint2_xpressor}). This zero-shot performance increase is solely due to
the improvement in the batch-correction score from using our \gls{VAE} method
(see Supplementary Figure~\ref{fig-s19-whisker-plot-of-open-problems-batch-integration-with-batch-correction-only-scores}). We then fine-tune the XPressor architecture alone
-- our XPEFT approach -- to further learn to remove batch effects and
predict expert-annotated cell-type labels. We add a Maximum Mean
Discrepancy (MMD) loss (see \hyperref[fine-tuning-task]{Methods})
that penalizes the distance between batch
elements\textsuperscript{67,68}. Doing so, we observe a jump in scIB
scores, especially in biological truthfulness, as measured by the scIB
metrics (see Figure 4C; Supplementary Figure~\ref{fig-s20-whisker-plot-open-problems-batch-integration-with-bio-conservation-only-scores}), making \gls{scPRINT}-2 the
best-performing method in the benchmark.

\subsubsection{scPRINT-2 generalizes to unseen cell embedding
tasks}\label{scprint-2-generalizes-to-unseen-cell-embedding-tasks}

We then wanted to push our analysis further and test the zero-shot
organism-level integration of \gls{scPRINT}-2 on organisms unseen during
training. Again, using our cat and tiger dataset presented in the
\hyperref[a-diverse-dataset-of-350-million-cells-pushes-generalization-to-unseen-organisms]{second
result section}\textbf{,} we saw that already, \gls{scPRINT}-2's general cell
embedding performs better than doing no correction and keeps lot of
biological truthfulness, as shown by the \gls{scIB} score of 0.44 vs 0.37 for
PCA (see Figure \ref{fig:scprint2_xpressor}, see Supplementary Figures~\ref{fig-s21-umap-of-scprint-2s-zero-shot-multi-species-expression-embedding-using-the-full-cell-embedding}, \ref{fig-s22-barplot-of-scib-score-on-scprint-2s-multi-species-integration}, see Supplementary Table~\ref{table-s4-detailed-scib-scores-on-the-unseen-species-integration-task}).
Then, as often, taking the cell-type-specific embedding further
increases the biological truthfulness to 0.49, mainly by generating a
more faithful biological representation, as reflected in the \gls{scIB} scores
(see Figure \ref{fig:scprint2_xpressor}, Supplementary Figures~\ref{fig-s22-barplot-of-scib-score-on-scprint-2s-multi-species-integration}, \ref{fig-s23-umap-of-scprint-2s-zero-shot-multi-species-expression-embedding-using-the-cell-type-cell-embedding}, Supplementary Table~\ref{table-s4-detailed-scib-scores-on-the-unseen-species-integration-task}). Again, using
XPEFT, we achieve a tremendous 0.60 \gls{scIB} score, placing us among the top
3 best-performing models in this category, behind SATURN and scGEN (see
Figure \ref{fig:scprint2_denoising}, Supplementary Figures~\ref{fig-s22-barplot-of-scib-score-on-scprint-2s-multi-species-integration}, \ref{fig-s24-umap-of-scprint-2s-multi-species-expression-embedding-post-finetuning-using-the-full-cell-embedding}, Supplementary Table~\ref{table-s3-detailed-scib-biological-conservation-scores-on-the-xenium-dataset}). We note that even in
this domain, many cell types didn't overlap across organisms. It is a
common behavior in this benchmark, and similar cell types now almost
overlap in the \gls{UMAP}, hinting at shared neighbors (see Figure \ref{fig:scprint2_denoising}, see
\hyperref[embedding-task]{Methods})\textsuperscript{69}.

Finally, we wanted to examine the model's ability not only to integrate
cellular profiles but also to generate entirely new ones at inference
time in a zero-shot manner by combining cell tokens (see Figure \ref{fig:scprint2_xpressor}). We
first approach it using a matched mouse-human multi-organ atlas from
Zhong et al.\textsuperscript{46}. We then generated cell embeddings for
all cells and computed an average ``human''-ness cell embedding using
the \emph{organism} embeddings of all human cells. We regenerate an
expression profile using 1. the human gene embedding and 2. the mouse
cell embeddings, replacing the organism cell embedding with the human
one (see Figure \ref{fig:scprint2_xpressor} and \hyperref[generative-task]{Methods}). We
thus generate a set of human-like cell expression profiles from mouse
expression profiles. Using the 5000 most variable orthologous genes, we
indeed observed a decrease in the Wasserstein-2 (W2) distance on this
counterfactual conversion to human (see Figure \ref{fig:scprint2_xpressor}, see
\hyperref[scrna-seq-datasets-distances]{Methods})\textsuperscript{70,71}.
Applying a similar approach, but this time to generate females from
males in the human dataset, we also notice a similar reduction in
expression W2-distance from 1076 to 938.

Looking at how cell expression patterns change after this transition, we
found that most of the top differentially expressed genes are the same
as those identified in the differential expression analysis of the real
human dataset (see Figure \ref{fig:scprint2_xpressor}, see Supplementary Figure~\ref{fig-s25-differential-expression-plot-of-the-human-vs-mouse-dataset-from-section-4}). Computing an
over-representation test, we observe a robust 58\% enrichment compared
to random, with more than half of the top differentially expressed genes
correctly predicted by \gls{scPRINT}-2 in both over- and under-expressed genes
(see Figure \ref{fig:scprint2_xpressor}, see Supplementary Figures~\ref{fig-s26-over-representation-plot-of-humanized-mouse-data-vs-real-mouse-data-compared-to-human}, \ref{fig-s27-over-representation-plot-of-female-like-male-data-vs-real-female-data-compared-to-male}). Looking at
\emph{Reactome\_2022} pathway enrichments, we see multiple pathways
related to immune system function, membrane-ECM (Extra-Cellular Matrix)
interactions, and tissue elasticity, as well as many other
molecular-level pathways (see Supplementary Figure~\ref{fig-s28-dot-plot-of-gene-set-enrichment-analysis-over-the-differential-expression-analysis-of-section-4}). These align with
previous analyses highlighting ECM and immune function differences
between human and mouse tissues\textsuperscript{72,73}.

Overall, we have shown that an entirely novel architecture and a set of
learning constraints enable \gls{scPRINT}-2 to generate high-quality
embeddings in a zero-shot manner. Thanks to its multi-organism training,
this can be extended to unseen species, while achieving even stronger
results with fine-tuning. We have also demonstrated how one can use the
\gls{scPRINT}-2's cell embeddings to generate counterfactual cellular
profiles. This makes it a strong contender for performing atlas-scale
analysis across tissues, diseases, and organisms, by learning to
disentangle each cell component. We will now focus on how other parts of the
models can be used to extract additional information.

\subsection{High-quality contextual gene representations from
scPRINT-2}\label{high-quality-contextual-gene-representations-from-scprint-2}

\subsubsection{scPRINT-2 has rich gene
embeddings}\label{scprint-2-has-rich-gene-embeddings}

\gls{scFM}s don't just provide cell-level embedding, they have also been used
to generate contextual gene-level embeddings given a cell's expression
profile or to predict gene-gene connections. The model' s
gene embeddings can be used for fine-tuning, such as to predict ATAC-seq
activities or gene essentiality\textsuperscript{1,2}. We investigate the
gene embeddings produced by \gls{scPRINT}-2 and then delve into how its gene
networks can be better extracted and assessed.

A good output gene embedding is also defined by the quality of its
input. With \gls{scPRINT}-2, we introduced a fine-tuning adapter layer on top
of ESM3's protein embeddings, jointly trained with the model (see \hyperref[methods]{Methods}).
This approach is one of the few that improve gene network inference
without decreasing any other metrics in our additive benchmark (see
Table \ref{tab:additive_benchmark}). It allows us to update gene representations during
pre-training while maintaining the ability to work with unseen
representations, e.g., from unseen species (see Figure \ref{fig:scprint2_genenetwork}).

It remains unclear, however, what the right approach is for selecting
output gene embeddings, with some heuristics proposing using the last or
second-to-last layer. Using our regular transformer model trained with
masking, we demonstrate that its output gene embeddings contain only
their own expression values (see Figure \ref{fig:scprint2_genenetwork}). However, when trained with
the Xpressor architecture, clusters of genes appear (see Figure \ref{fig:scprint2_genenetwork}).
This is sensible because Xpressor forces gene embeddings to be rich in
meaning, as the compression block must query them. We have, however,
noticed that for regular models that are not fully trained (only up to
20 epochs), the output gene embedding still contains some input ESM3
features (see Supplementary Figure~\ref{fig-s29-output-gene-embedding-for-a-non-fully-trained-model-without-xpressor-architecture}). The number of enriched pathways in its
output gene embedding cluster is still significantly less than for
\gls{scPRINT}-2's XPressor architecture (see Figure \ref{fig:scprint2_genenetwork}; see
\hyperref[assessment-of-gene-output-embeddings]{Methods})

\begin{figure}[H]
    \centering
    \includegraphics[width=0.9\textwidth]{figures/scprint2/image3.png}
    \caption[Presentation of the ESM3 fine-tuning and gene network study]{(a) Illustration of fine-tuning of ESM3 while training \gls{scPRINT}-2 using an adaptor layer. (b) Comparison of gene output-embeddings for a random cell in a model with the XPressor architecture and a model trained without. (c) On the side, the average number of pathways shown to be enriched in the gene output embedding clustering of each method using three main pathway databases. The number below is on the non-fully trained regular transformer; otherwise, no pathways are enriched (see Methods). (d) Comparison of ground truth networks' overlap between cellmap, the human interactome, and genome-wide perturb-seq. (e) Benchmark over six ground truth gene networks of scPRINT-1's gene networks with its extraction method and \gls{scPRINT}-2's gene networks with its extraction method, over nine different human cell types from the same dataset. (f) Comparison of the top-30 hub nodes on both gene networks. Arrows link similar genes, and colors represent similar gene groups. (g) Subset of a gene network generated by \gls{scPRINT}-2 seeded at FTL1, on human macrophage cells, and on mouse macrophage cells, edge color represents the RoseTTAFold2-PPI scores for these connections, grey means no score was computed. The AlphaFold-Multimer structure and amino-acid distance map are provided for the star-marked connections. Source data are provided as a Source Data file.}
    \label{fig:scprint2_genenetwork}
\end{figure}

\subsubsection{extracting gene networks from
\gls{scPRINT}-2}\label{extracting-gene-networks-from-scprint-2}

Thanks to the transformer architecture, one can go beyond gene output
embeddings to examine gene-gene interactions via the model's attention
layers. Following the tests reported in Kalfon et al., we observed, on
average, no dramatic performance gains across the methods we tested (see
Table \ref{tab:additive_benchmark}). An issue we noticed is that the problem is not well-defined.
Indeed, the ground truths widely disagreed with one another (see Figure
\ref{fig:scprint2_genenetwork}; see Methods).
Between the genome-wide perturb-seq (gwps) ground truth and omnipath,
only 800 gene-gene connections were in common over the hundreds of
thousands that each contained. This suggests that diversity of ground
truth will be key to showcasing the breadth of potential gene-gene
connections in the cell.

We thus gathered a new set of ground truth gene networks (GN)s from
recent works. Our first approach was to use protein-binding datafrom
AP-MS experiments within the O2US cell line, called the
\emph{cellmap}\textsuperscript{74} (see
\hyperref[gene-network-task]{Methods}). Additionally, thanks to
protein structure models, we are now able to compute putative
interactions across millions of protein pairs; a first version of this
analysis has been defined in the human \emph{interactome} (see
\hyperref[gene-network-task]{Methods}). But here again, the
disagreement was significant, with only \textasciitilde1-4\% of the
connections in each ground truth being found in another, and no
connections were reliably found across all five ground truths (see Supplementary
Figure~\ref{fig-s30-venn-diagram-of-the-different-ground-truth-gene-networks}).

Acknowledging these disagreements, we benchmarked them against nine
human cell types from the same dataset using scPRINT-1 and \gls{scPRINT}-2. We
use a gene network extraction method that is more computationally
demanding but biases the network towards co-expressed genes (see
\hyperref[extracting-meta-cell-gene-networks-from-attention-matrices]{Methods}).
We see that \gls{scPRINT}-2's performance was often greater or similar across
all benchmark networks, as indicated by the odds-ratio measures (see
Figure \ref{fig:scprint2_genenetwork}; see Methods).
We did not see a similar trend, however, on \gls{AUPRC} (see Supplementary Figure~\ref{fig-s31-whisker-plot-of-auprc-ratio-scores-for-scprint-1-and-scprint-2}). This suggests that our method is more accurate for its top-K
connections. Indeed, the strongest human interactome connections were
overrepresented in \gls{scPRINT}-2, more so than in scPRINT-1.

\subsubsection{cross-organism gene network
analysis}\label{cross-organism-gene-network-analysis}

To continue on our cross-organism analysis, we also aimed to further
characterize some of the genes observed in our previous human/mouse
datasets by interrogating the cell-specific GN identified by scPRINT-2
in \emph{Macrophage} cells from both mammals. Looking at their hub
nodes, we see that many are common and represent key conserved cell
immune pathways, such as \emph{feroptosis}, \emph{vitamin B12}, and
\emph{Pathogen Phagocytosis Pathways} (\emph{WikiPathway\_2023\_Human}),
with genes like \emph{C1Qs, RPs, ALB, and APOE}\textsuperscript{75,76}.
These mainly relate to the macrophage' s internal
machinery, which is designed to eat and destroy pathogens. Other genes
were clear markers of macrophages (\emph{CD74;LYZ) and/or immune cells
(HLA-DRA, B2M) or their pathways, such as interferons alpha/gamma and
MHC Class II (MSigDB\_Hallmark\_202076--78)}. Interestingly, these
networks share only 30\% similarity when considering the top 20
connections for each gene. But what seemed like differences in
connections and top 50 hub genes tended to disappear after thorough
analysis, such as with the Ribosomal proteins, which are related in the
kinds of pathways they are part of, or in their relationships in the
PPI\_Hub\_Proteins database\textsuperscript{77} (see Figure \ref{fig:scprint2_genenetwork}).

We then extracted a subset of the macrophage networks, seeded at the
\emph{FTH1} gene, for both organisms, focusing on the top 15 connected
nodes and their top 60 edges (see Figure \ref{fig:scprint2_genenetwork}; see
\hyperref[plotting-gene-sub-networks]{Methods}). We observed a set
of hub genes in both subnetworks, with some genes being shared between
human and mouse. Interestingly, these hub genes had more interactions in
the human interactome ground truth than non-hub genes. We also noticed
that the ``hub-ness'' of the subnetworks can be very variable and seems
to depend on the ``seed'' gene (see Supplementary Figure~\ref{fig-s32-additional-scprint-2-generated-gene-network-computed-from-cdc45}).

By overlaying the human interactome ground-truth values on our
subnetworks, we found that only a small subset of connections was marked
as valid (i.e., score above 0.6) in the ground truth (see Figure \ref{fig:scprint2_genenetwork}). In
the mouse \emph{Macrophage} subnetwork, almost no connections were
recovered, but this may be explained by the fact that the ground truth
is the ``human'' interactome, computed using human proteins rather than
mouse proteins. We thus wondered whether we could use scPRINT-2 to
cross-validate the interactions present in this ground truth. Indeed, we
know that the human interactome values are not directly computed from
AlphaFold-multimer's interaction probability (ipTM); they come from a
simpler model called ``RoseTTAFold2-PPI''. Testing a couple of
connections predicted to be low ipTM by RoseTTAFold2-PPI but found by
scPRINT-2, we readily identified two: HLA-DRA/CD74 and B2M/B2M, which,
when passed to AlphaFold-Multimer, indeed formed an interaction with an
ipTM of more than 0.6. This showcases the potential of scPRINT-2 in this
domain and future directions for GN inference.

We have seen here how scPRINT-2's output gene embeddings and attention
matrices can be used to extract meaningful biological insights and drive
hypothesis generation in a cell-to-cell, state-specific manner. These
outputs can also be used for fine-tuning purposes and in explainable
AI-driven analysis. We also pushed our GN analysis further, defining
additional benchmarks and a more powerful GN extraction mechanism. We
demonstrated cross-species analysis and presented the tantalizing
possibility of merging foundation models at different scales, including
ESM3 fine-tuning, AlphaFold Multimer, RoseTTAFold2-PPI, and scPRINT-2.
While these are just examples, they demonstrate what aggregating
multiple bodies of evidence across scales can achieve for genetic
interaction predictions. A first step towards using \gls{scFM}s, protein
Language Models, and structural models in coordination, to shed light on
the cellular machinery.

\section{Discussion}\label{discussion}

In this work, we present a gymnasium of tasks to benchmark \gls{scFM}s in
multiple contexts. Together with an efficient and reproducible pipeline,
we test the benefits of 42 different parts of \gls{scFM}s structures,
encoding, and training. In this additive benchmark, 12 of these are our
own contributions to \gls{scFM}s, including GNN-based expression encoding,
cross-foundation model fine-tuning, sub-quadratic attention mechanisms,
and rich losses. This massive benchmark is the first of its kind for
\gls{scFM}s and assesses four different tasks. It allowed us to identify
bottlenecks and limitations, issues that we solved in subsequent
analysis. Indeed, future benchmarks will benefit from using more diverse
datasets, tasks, and ground truths.

We have also presented the largest pre-training database to date,
encompassing more organisms, conditions, and data modalities. We have
seen that, while more work is needed to obtain higher-quality,
well-annotated datasets, our dataloader and preprocessing pipeline have
made the most of this vast database.

Using the best feature combinations from our additive benchmarking, we
build and train a next-generation cell Foundation Model, scPRINT-2. We
demonstrate that, although currently 5 times smaller, scPRINT-2
outperforms scPRINT-1 across all benchmarks tested. On denoising,
scPRINT-2 becomes state-of-the-art, and with our fine-tuning approach,
it also outperforms every other method on the batch-correction and
classification tasks of the open-problem benchmarks.

We then challenge scPRINT-2 on tasks of high relevance for cellular
biology, highlighting some pitfalls in current benchmarks. We show that
scPRINT-2 acquires generalizable abilities across unseen modalities and
organisms, while remaining consistent in its predictions. We demonstrate
it across many tasks, including cross-organism integration, unseen gene
imputation, and counterfactual reasoning.

Finally, we present tools for easily extracting labels, cell-specific
gene embeddings, imputing gene expression, performing gene network
inference, and working with organisms unseen during pre-training. We
believe our results demonstrate many domains where \gls{scFM}s might
confidently replace approaches that rely on heuristics, atlases, and a
variety of tools and packages. However, much work remains.

Current ground-truth cell annotations are cluster-based and obfuscate
the complexity of cellular states by inherent clustering biases. Batch
correction metrics are similarly biased, and top scores can be easily
gamed; gene network ground-truths are not cell type specific and likely
filled with false negatives. Data diversity and quality are the
principal pre-training bottlenecks, and efforts will be needed to
improve foundation models. Many other key modalities, such as measuring
time and perturbation effects, remain scarce. They will become
increasingly helpful for enriching the future comprehensive benchmarks
of next-generation cell foundation models.

Our analysis and contributions highlight powerful features of \gls{scFM}s and
provide guidance for designing benchmarks that better highlight their
strengths and weaknesses. scPRINT-2 presents a direction for future
improvements, with more specialized architectures and using a
combination of biological FMs working jointly across modalities and
scales. This next-generation \gls{scFM} is a step forward in the design of AI
for cell biology.

\section{Methods}\label{methods}

We present an additive benchmark with over a dozen contributions to the
pre-training tasks, losses, and architecture of single-cell foundation
models. Along with it, \textbf{scPRINT-2} (pronounced ``sprint''), a
next-generation model trained on the best-performing contributions. We
analyze its out-of-distribution generalization and present methods for
querying and fine-tuning it to solve various tasks. We will go through
the specific techniques that made it possible.

\subsection{Additive benchmark}\label{additive-benchmark-1}

We now describe in matched order with respect to Table 1, the methods
behind the multiple contributions tested in our additive benchmark (see
\hyperref[decoding-the-impact-of-a-foundation-models-architecture-through-an-additive-benchmark]{Results
section 1}). We bolded the ones that are further defined in the
methods. In this benchmark, we are using and testing the:

\begin{enumerate}
\def\labelenumi{\arabic{enumi}.}
\item
  \textbf{``base model'',} every subsequent element is applied to the
  base model
\item
  ``medium model'', larger base model, see the
  \hyperref[base-model-and-training]{\underline{base model section}}
\item
  ``negative control'', untrained base model
\end{enumerate}

Architecture

\begin{enumerate}
\def\labelenumi{\arabic{enumi}.}
\setcounter{enumi}{3}
\item
  ``no dropout'', where we remove the dropout initially applied in the
  base model
\item
  ``large classifier'', where the classifier sizes are increased from
  {[}input - output{]} in the base model to {[} input - 256 - output{]}
\item
  ``MVC'', where we replace the base model's decoder with the cell
  embedding's MVC approach of scGPT\textsuperscript{1}
\item
  ``no decoders/generation'', where we removed the base model's decoder,
  getting a masking+classification only pre-training
\end{enumerate}

Data

\begin{enumerate}
\def\labelenumi{\arabic{enumi}.}
\setcounter{enumi}{7}
\item
  replacing our pre-training dataset with ``Tahoe''\,'s 100M dataset
\item
  Chan Zuckerberg Institute (``CZI'')'s cellxgene database (version
  2024)
\item
  replacing our pretraining dataset with ``CZI + Tahoe'' with Tahoe's
  100M database
\item
  replacing our pretraining dataset with ``all databases'', both CZI,
  Tahor, and Arc's scBasecount\textsuperscript{26,27}
\item
  replacing our pretraining dataset with ``only 200 random'' human
  datasets
\item
  replacing our sampling with a ``sampling without replacement''
\item
  \textbf{replacing our sampling with ``cluster-based sampling only''}
\item
  \textbf{adding ``meta-cell'' during pre-training}
\end{enumerate}

Attention

\begin{enumerate}
\def\labelenumi{\arabic{enumi}.}
\setcounter{enumi}{15}
\item
  replacing FA3 with ``softpick'' attention, using the approach of Zuhri
  et al.\textsuperscript{57}
\item
  replacing FA3 with ``hyper''-attention, using the approach of Han et
  al.\textsuperscript{56}
\item
  replacing self-attention with \textbf{``criss-cross'' attention
  layers}
\item
  \textbf{adding ``XPressor'' layers}
\end{enumerate}

Losses

\begin{enumerate}
\def\labelenumi{\arabic{enumi}.}
\setcounter{enumi}{19}
\item
  \textbf{adding ``contrastive learning''}
\item
  \textbf{adding ``elastic cell similarity''}
\item
  \textbf{``no embedding independence loss'', removing the embedding
  independence loss}
\item
  replacing the \gls{ZINB} loss with Mean Squared Error (``MSE'')-loss
\item
  \textbf{replacing the \gls{ZINB} loss with ``ZINB+MSE'' loss}
\item
  \textbf{adding a ``VAE compressor'' loss to the Base model}
\end{enumerate}

\textbf{\hfill\break
}Tasks

\begin{enumerate}
\def\labelenumi{\arabic{enumi}.}
\setcounter{enumi}{25}
\item
  \textbf{adding ``variable context length'' and a larger context}
\item
  \textbf{replacing masking with a Transcription Factor ``(TF)-masking''
  task}
\item
  replacing masking with ``denoising'', using the approach in scPRINT,
  with a random level of denoising (see below)
\item
  ``no classification'', removing the classification pre-training task
\item
  \textbf{adding an ``adversarial classifier''}
\end{enumerate}

Input

\begin{enumerate}
\def\labelenumi{\arabic{enumi}.}
\setcounter{enumi}{30}
\item
  replacing log1p normalization with ``sum normalization'' where each
  expression profile is normalized to sum to 10,000
\item
  ``no random level of denoising'' where we remove the random level of
  denoising, see the
  \hyperref[denoising-pre-training-task]{\underline{denoising section}}
\item
  \textbf{where we replace the expression encoder with a Graph Neural
  Network (``GNN'') encoder}
\item
  where we replace the continuous expression encoder with a ``binning''
  version, following the approach of scGPT\textsuperscript{1}
\item
  where we are ``using only expressed genes'', as in scGPT and
  geneformer
\item
  ``without using gene location'', removing the gene location
  information in the input tokens.
\item
  ``learn gene embedding'' where we replace the ESM3 gene embedding with
  learnt embeddings, as in scGPT and Geneformer.
\item
  \textbf{replacing the ESM3 gene encoder with a ``fine-tuned ESM3''
  gene encoder}
\end{enumerate}

The full training traces of the entire additive benchmark are available
on weights and biases:

\noindent\url{https://wandb.ai/ml4ig/scprint_ablation/reports/scPRINT-2-additive-benchmark--VmlldzoxNTIyOTYwNA}

\subsubsection{Base model and training}\label{base-model-and-training}

The additive benchmark is performed on a small model with 18.2M
parameters, an embedding dimension of 256, and 8 layers and 4 heads. The
model trains for 20 epochs of 20,000 batches of 64 cells per batch.
Validation is performed on 10,000 minibatches. We otherwise use the same
optimizer and hyperparameters as for scPRINT-2 (see
\textbf{pre-training} in \hyperref[pre-training]{Methods})

Gene expression is encoded using ESM3 embedding, with gene location and
MLP-based expression encoding added, as described by Kalfon et al. The
output is decoded using an \gls{MLP} that takes the output embeddings and
depth information, then outputs a scalar expression value.

The base model is trained on CZI's cellxgene census dataset, version
2024 (compared to 2022 in Kalfon et al.). The pre-training task uses a
30\% gene expression mask with an \gls{MSE} loss (as is common for BERT-like
encoder transformers)\textsuperscript{1,2,7}. The Base model also uses a
multi-cell-token generative loss as described in Kalfon et
al.\textsuperscript{3}. It also performs matched multi-class
hierarchical classification, as defined below (see \textbf{Hierarchical
classifier} in the
\hyperref[hierarchical-classifier-loss]{Methods}\textbf{).}
Finally, it also uses a dissimilarity loss between each of our cell
embeddings for a given cell (see
\hyperref[embedding-independence-loss]{\underline{embedding independence in
Methods}}).

Each of these decisions is assessed within our additive study.

We pre-train the base model 6 times across multiple seeds to generate
error bounds. We train using Flash-Attention-3 on 1 H100 GPU, each
training of 20 epochs taking roughly 2 days. Some runs were done on
A100s and V100s; we thus had to rescale the time duration for some of
these runs.

Some additive study runs use denoising as a training strategy or larger
context lengths when it seemed likely that this would best highlight the
abilities and shortcomings of the benchmarked element.

The \textbf{medium model} size uses an embedding dimension of 512, with
16 layers and 8 heads.

The \textbf{negative control} is a model that was not trained at all.

\subsubsection{Weighted sampling}\label{weighted-sampling}

The goal of weighted random sampling is to de-bias regular random
sampling of cells in contexts where many cells have similar profiles and
expression patterns, while others are rare cell types.

We use weighted random sampling on our training data based on all the
different class values we have to predict. We use a factor of \(S_{1}\),
meaning the rarest elements will, on average, be sampled only \(S_{1}\)
times less than the most common ones. The sampling factor used for each
group is then\(\frac{S_{1}}{c + S_{1}}\) , instead of \(\frac{1}{c}\),
where \(c\) is the number of cells in each group.

\subsubsection{Cluster-weighted
sampling}\label{cluster-weighted-sampling}

The goal of cluster-weighted sampling is to improve weighted sampling in
the condition where cell-type annotations are poor or non-existent.

For cluster-weighted sampling, we simply use the labels obtained by
applying Leiden clustering to the K-NN graph of cells for each dataset
during preprocessing. We used a resolution of 1 and 15 neighbors. We
merge clusters if their centroid correlation exceeds a threshold (here
94\%). This cluster label is then treated similarly to other labels,
such as \emph{cell\_type}, \emph{sequencer}, etc.

In this context, within datasets that lack information about tissue of
origin or sequencer, or that belong to the same categories, cells from
cluster 1 will be sampled with equal weight from those datasets. The
sampling is not dataset-specific. This decision arises because most
datasets contain some information about their tissue of origin or
disease, and cluster sizes of data from the same tissue/disease often
represent similar cells.

This can be applied to any dataset for training models.

\subsubsection{Depth-weighted sampling}\label{depth-weighted-sampling}

The goal of depth-weighted sampling is to sample cells with higher
quality, in terms of the number of genes expressed, more often.

For depth-weighted sampling, we scale each cell' s
sampling probability by its non-zero (nnz) gene count. Similarly, we
scale this value, but this time we apply a sigmoid function beforehand
to reduce the impact of extreme values.

Algorithm: scale\_nnz (1)

Input:

- midpoint: 2000

- steepness: 0.003

- scale: 1000

Output: unormalized sampling probabilities

\# Apply sigmoid transformation

sigmoid\_values = 1 / (1 + np.exp(-steepness * (nnz - midpoint)))

\# Then scale to {[}1, scale{]} range

return 1 + ((scale - 1) * sigmoid\_values)

The values shown for Input were the ones we used across our research and
were selected manually.

This can be applied to any single-cell dataset for training models.

\subsubsection{Multi-cell sampling}\label{multi-cell-sampling}

For all our datasets, our preprocessing pipeline computes a K-NN graph
from the \gls{PCA} of the scaled, log-transformed expression data. For each
sampled cell, scDataloader also retrieves its k-NN cell ID and loads
them, along with their distance information. Here, we set K to 6 and the
PCA components to 200.

We set the number of \gls{PCA} components to 200 to retain as much information
as possible, while accounting for rare cells whose expression might have
only a small impact on the first \gls{PCA} components.

We set K to 6 to balance the computational resources required to sample
6 times more cells per minibatch with the need for enough neighboring
cells. Indeed, these computational resources are more prevalent for
smaller models that perform fast iterations across many cells than for
larger models. 6 neighbors per sampled cell was our limit for a small
foundation model like scPRINT-2. We also note that there is likely a
rapid diminishing returns beyond 6 to 15 cells for most datasets as we
start sampling more often cells that are less similar to the center one.

During scPRINT-2 training, we select 0 to 6 neighbors per minibatch, so
the model learns to use a variable number of cell neighbors.

\subsubsection{GNN Expression encoder}\label{gnn-expression-encoder}

The goal of the \gls{GNN} expression encoder is to increase the information
the foundation model can obtain from 1 cell to a set of neighboring
cells, thereby dramatically reducing input noise.

The \gls{GNN} takes multiple expression values as input, optionally along with
corresponding cell-cell distances, and returns a vector encoding this
information. Both continuous and \gls{GNN} encodings can be configured to
receive either logp1-transformed expression data, sum-normalized
expression, or both. The \gls{GNN} follows the DeepSet\textsuperscript{78}
implementation:

\begin{equation}
E_{j} = \text{DeepSet}(x_{ij}, n_{ij}, d) = \phi_{1}(\phi_{2}(x_{ij})||\phi_{3}(n_{ij},d_{ij}))
\end{equation}

Where:

\begin{itemize}
\item
  \(x_{j}\) is the center cell's expression for the gene \(j\)
\item
  \(n_{ij} \in \Re^{k}\) is the K nearest neighbor cell's expression for
  gene j and cell i
\item
  \(d_{ij} \in \Re^{k}\) is the distance of each neighboring cell to the
  center one
\item
  \(\phi_{i}\) are MLPs.
\item
  \(||\) is the concat operation
\end{itemize}

We selected K to be a random number between 0 and 6 during training and
6 at inference.

\subsubsection{ESM3 fine-tuning
gene-encoder}\label{esm3-fine-tuning-gene-encoder}

The goal of ESM3 fine-tuning is to get the best of both worlds between
learning token features from the data and using learnt protein
representations from a pLM as a prior.

We encode/tokenize gene IDs using ESM3\textsuperscript{79}. The mapping
process happens in the following way:

\begin{itemize}
\item
  A gene name is mapped to its canonical protein name using Ensembl114.
\item
  We recover the protein sequence of the protein using Ensembl
\item
  We use the protein sequence to generate an embedding using ESM3 by
  averaging all its amino-acid output embeddings.
\end{itemize}

For the fine-tuning part, we reuse the fine-tuning approaches presented
in Kalfon et al., which place an additional adapter layer after
mean-pooling and before feeding the protein representation to the model.
Interestingly, using gene expression as a further signal to the adaptor
layer often led to training instability.

\subsubsection{Biased attention}\label{biased-attention}

The goal of biased attention is to orient our attention matrix towards
genetic interaction priors to improve learning and the model's
biological fidelity.

We leveraged the Rcistarget computation and ranking of the human genome
for 10kb down- and upstream of each target gene\textsuperscript{80},
available at the Aerts Lab cistarget databases\footnote{\url{https://resources.aertslab.org/cistarget/databases/homo_sapiens/hg38/refseq_r80/mc_v10_clust/gene_based/hg38_10kbp_up_10kbp_down_full_tx_v10_clust.genes_vs_motifs.rankings.feather}}.
Using this information, we generate a weight matrix \(M\) that links
each motif-defined TF to its target genes.

Given this matrix, we bias the attention matrix for all heads and layers
using the attn\_mask parameter of the
torch.nn.functional.scaled\_dot\_product\_attention function.

It appears in the attention computation like so:
$\text{softmax}\left(\frac{QK^{\top}}{\sqrt{d}} + M\right)V$ where $M$ is the
attn\_mask matrix and is real-valued.

\subsubsection{Criss-Cross attention}\label{criss-cross-attention}

The goal of criss-cross attention is to create an efficient attention
mechanism by learning, in context, a factorisation of each attention
matrix.

In criss-cross attention, we replace the self-attention mechanism with a
double cross attention between the \(N\) input elements and \(M\) latent
tokens (see Supplementary Figure~\ref{fig-s18-illustration-of-the-similarity-and-dissimilarity-based-contrastive-losses-used-in-scprint-2}). This is thus replacing a \(N^{2}\)
computation with a \(2NM\) one, hence going below the quadratic
bottleneck of attention. This bears resemblance to the ISAB
architecture, XPressor, and
perceiverIO\textsuperscript{58,63--65,81,82}. M, in our case, is set to
10: our 9 predicted classes plus an additional token.

Effectively, the \(M\) latent tokens are learnt at the first layer of
the models. At the same time, they could also be generated from a
sketching or principal components analysis (PCA) of the input tokens.
They also get updated during the attention computation, so that at the
second layer.

We replace the traditional attention computation
$X_{l + 1} = \text{Attention}(X_{l},X_{l},X_{l}) + X_{l}$, where Attention
takes as input the Query, Key, Value elements, with:
\begin{equation}
\text{Attention}(X_{1},X_{2},X_{3}) = \text{softmax}\left(\frac{QK^{\top}}{\sqrt{d}}\right)V
\end{equation}

With

\begin{itemize}
\item
  \(Q = X_{1}W_{Q},\ K = X_{2}W_{K},\ V = X_{3}W_{V}\)
\end{itemize}

\begin{itemize}
\item
  \(W_{Q},\ W_{K},W_{V}\  \in \ \Re^{dxd},\ X_{} \in \ \Re^{Nxd}\)
\end{itemize}

In self-attention \(X_{1} = X_{2} = X_{3}\)

In Criss-Cross attention, the algorithm becomes:

\(V_{l + 1} = Att(V_{l},X_{l},\ X_{l}) + V_{l}\) (3)

for the latent update and

\(X_{l + 1} = Att(X_{l},\ V_{l},\ V_{l}) + X_{l}\) (4)

for the main update

with \(X_{l} \in \Re^{Nxd}\) the main embeddings and
\(V_{l} \in \Re^{Mxe}\)the latent embeddings

\subsubsection{XPressor model}\label{xpressor-model}

The goal of the XPressor architecture, as presented in Kalfon et al., is
to replace and generalize the class-pooling of other transformer models
and the bottleneck learning of scPRINT. This makes the model more
powerful at encoding cell-level features while also separating
cell-level tokens from gene-level tokens. Finally, it enables a new mode
of Parameter-Efficient Fine-Tuning. This bears similarities to the ideas
presented in criss-cross attention above.

The \textbf{Xpressor} block uses as input a set of learned latent tokens
\(T\). It then performs cross-attention between the last layer of the
gene embeddings and the latent tokens. The goal is for the
\textbf{Xpressor} layers to be of smaller dimensions and context size
than the main transformer layers, such that we end up with \(C_{j}\) a
set of \(n\) tokens of dimension \(d_{t}\) generated from the encoded
gene expression and ID matrices \(E_{j}\) , and \(G\). Where \(G\ \)and
\(E_{j}\ \)are sets of \(m\) tokens of size \(d_{c}\) representing the
IDs of the genes and their corresponding expression in cell \$j\$,
respectively, where \(d_{c} < d_{t}\) and \(n < < \ m\):

\(O_{j} = Transformer(E_{j},\ G)\) (5)

\(C_{j} = Xpressor(O_{j},T)\) (6)

For a cell \(j\), with the \textbf{Xpressor} being initialized with a
learned set of input cell tokens, and \(C_{j}\) being the cell tokens
associated with the input \(E_{j}\).

The \textbf{Transformer} and \textbf{Xpressor} are both transformers
with \(l_{1}\) and \(l_{2}\) layers, respectively. Indeed, we have
designed both layers to contain a cross-attention architecture (see
Figure 4A, Supplementary Figure~\ref{fig-s1-illustration-of-the-full-scprint-2s-architecture-input-and-output}) such that we can also do:

\(O_{j} = Transformer(C_{j},G)\), (7)

With \(O_{j}\) the output of the \textbf{Transformer} when using the
\textbf{Xpressor} representation as input.

We add an optional \gls{MLP} after cross-attention to transform the embeddings
before the self-attention round. In our example, the decompression is
performed using gene ID tokens as input only. These tokens remain the
same for all cells of a given organism and thus do not depend on \(j\).
In the context of protein language models, for example, this would be
replaced by positional tokens.

As shown in Supplementary Figure~\ref{fig-s1-illustration-of-the-full-scprint-2s-architecture-input-and-output}, the \textbf{Transformer} blocks are
applied twice. The first application serves as an ``encoder'', using
only self-attention, while the \textbf{Xpressor} and the second
application of the \textbf{Transformer} blocks act as ``decoders''. We
follow these definitions from the original "Attention is All You Need"
paper\textsuperscript{83}. It should be noted that, in our case,
cross-attention is performed before self-attention.

Related ideas have also been explored in the NVEmbed paper, where the
authors propose a cross-attention-based method to update tokens using
"latent" tokens and some additional prompting
tricks\textsuperscript{66}.

XPressor can be applied during pre-training or fine-tuning to replace
mean-max-class pooling in Foundation models.

\subsubsection{VAE-based compressor
model}\label{vae-based-compressor-model}

The goal of the VAE-based compressor is to reduce information sharing
between output embeddings by penalizing the amount of information each
embedding stores (see Figure \ref{fig:scprint2_xpressor}, Supplementary Figure~\ref{fig-s1-illustration-of-the-full-scprint-2s-architecture-input-and-output})\textsuperscript{84}.

Each VAE-based compressor is explicitly applied to a cell embedding,
compressing it into a relevant latent dimension. It has a 2-layer MLP
encoder and a 2-layer \gls{MLP} decoder. In cases where only a small set of
possible elements exists, such as in sex embeddings or cell culture, one
can use the Finite Scalar Quantization (FSQ)-VAE\textsuperscript{85}.

FSQ-VAE discretizes each latent dimension \textbf{independently}.
Specifically, the encoder outputs \(d\) values, each constrained to lie
within a bounded range (e.g., {[}-1, 1{]}). Each dimension is then
quantized into one of \(M\) discrete levels within that range (in our
case 2). This dimension-wise quantization can be implemented as either a
hard nearest-bin assignment or a differentiable approximation thereof.
Because FSQ enforces scalar-level discretization, it provides a simpler
and more fine-grained alternative to VQ' s vector-level
codebook approach, while still offering strong regularization of the
latent space.

In our case, all \gls{VAE}s with fewer than 8 latent dimensions used the
FSQ-VAE approach.

It can be applied on top of any output embedding at pre-training or
fine-tuning.

\subsubsection{\gls{ZINB}+MSE loss}\label{zinbmse-loss}

The goal of the ZINB+MSE loss is to make the model' s
expression-level prediction as precise as possible (thanks to the MSE)
while preserving the \gls{ZINB}' s expressivity and uncertainty
estimation.

scPRINT-2 uses a novel expression decoder for foundation models that
outputs the parameters of a zero-inflated negative binomial (\gls{ZINB})
distribution for each gene \emph{i} in cell \emph{j}. The \emph{\gls{ZINB}}
distribution is defined as

\begin{equation}
X\sim \gls{ZINB}(\mu,\ \theta,\ \pi)
\end{equation}

Where the parameters \(\mu,\ \theta,\ \pi\) are obtained from a
multi-layer perceptron (MLP) applied to the expression embeddings
outputted by the transformer model at its last layer (e), which are the:

\begin{equation}
\mu,\ \theta,\ \pi\  = \ MLP(e\ ||\ d)
\end{equation}

The \gls{MLP} is a two-layer neural network with dimensions {[}\emph{d+1, d},
3{]}, where \textbar\textbar{} denotes the concatenation operation.

Compared to scVI, where the overdispersion parameter \(\theta\) is
learned for each gene, we make scPRINT-2 output it together with
\(\mu,\ \pi\) (see Supplementary Figure~\ref{fig-s13-illustration-of-scprint-2s-generative-imputation-mechanism})

Effectively, the model learns that dispersion may vary across genes,
sequencers, cell types, and sequencing depths.

In addition, the loss adds an \gls{MSE} term computed from the \(\mu\) and
\(\theta\) output of the MLP, comparing for a gene \(i\),
\(e_{i}\  = \ \mu_{i} \times (1 - \sigma(\pi_{i}))\) to the
logp1-transform of the expression using mean-squared-error.

Where \(e_{i}\) is the predicted expression of gene \(i\) and \(\sigma\)
is the sigmoid
function:\begin{equation}
L_{MSE} = \frac{1}{n} \sum_{i = 1}^{n}(e_{i} - \log_2(x_{i} + 1))^{2}
\end{equation}

The zinb+mse loss is the addition of both losses with a scale parameter,
here:

\begin{equation}
L_{ZINB + MSE} = L_{ZINB} + 0.{5 \times L}_{MSE}
\end{equation}

This loss comes as a replacement for the classical \gls{MSE} or \gls{ZINB} in
\gls{scRNA-seq} models.

\subsubsection{\texorpdfstring{Embedding contrastive loss
}{Embedding contrastive loss }}\label{embedding-contrastive-loss}

The goal of this contrastive loss is to remove some batch effect by
pushing cell embeddings obtained from the expression profile after
different perturbations to be more similar to each other than they are
from cell embeddings of other cell profiles, using the
InfoNCE\textsuperscript{88} loss:

Algorithm: L\_contrastive (2)

Input:

- x: embeddings of cells post perturbation A {[}batch\_size ×
feature\_dim{]}

- y: embeddings of the same cells post perturbation B {[}batch\_size ×
feature\_dim{]}

- temperature: scaling parameter τ = 0.3

Output: contrastive loss value

1. // Compute similarity matrix

S ← cosine\_similarity\_matrix(x, y) / τ

Where S{[}i,j{]} = (x{[}i{]} · y{[}j{]}) /
(\textbar\textbar x{[}i{]}\textbar\textbar{}
\textbar\textbar y{[}j{]}\textbar\textbar{} τ)

2. // Create positive pair labels

labels ← {[}0, 1, 2, ..., batch\_size-1{]}

3. // Compute cross-entropy loss

loss ← cross\_entropy(S, labels)

Which expands to:

loss ← -Σᵢ log(exp(S{[}i,i{]}) / Σⱼ exp(S{[}i,j{]}))

Return loss

This loss can be added to any \gls{scFM}s at pre-training or fine-tuning (see
Supplementary Figure~\ref{fig-s18-illustration-of-the-similarity-and-dissimilarity-based-contrastive-losses-used-in-scprint-2}).

\subsubsection{Elastic cell similarity
loss}\label{elastic-cell-similarity-loss}

The goal of this loss is to reduce batch effects by pushing cells that
are similar to become more similar and cells that are dissimilar to
become more dissimilar\textsuperscript{1}.

We implement the \textbf{cell similarity loss} of scGPT, where, given
cell embeddings $e \in \mathbb{R}^{m \times d}$, where $m$ is the number of cells
and $d$ is the embedding dimension:
\begin{equation}
L_{similarity} = \frac{1}{m(m - 1)}\sum_{i \neq j} 1 - (\max(0, \hat{e}_i^\top \hat{e}_j) - \tau)^2
\end{equation}
Where:

$\hat{e}_i = e_{i}/\|e_{i}\|_2$ is the L2-normalized embedding of the cell
$i$

\(\tau\) is the similarity threshold (default 0.3)

\(m(m - 1)\) is the number of off-diagonal pairs,

This loss can be added to any \gls{scFM}s.

\subsubsection{Embedding independence
loss}\label{embedding-independence-loss}

The goal of the embedding independence loss is to push the different
class-level embeddings of a cell to encode distinct information by
making them orthogonal (see Supplementary Figure~\ref{fig-s18-illustration-of-the-similarity-and-dissimilarity-based-contrastive-losses-used-in-scprint-2}).

Implementing a set of disentangled embeddings is not straightforward. In
our case, we push the embeddings to be as different from one another as
possible, with an \textbf{independence loss} defined as

\begin{equation}
L_{independence} = \frac{1}{m^{2}}\sum_{i = 1}^{m}\sum_{i'}^{m}1 - \cos(e_{i},e_{i'})
\end{equation}

where \(e_{i}\) and \(e_{i'}\) are the cell embeddings, \emph{m} is the
minibatch size, and \emph{cos} denotes the cosine similarity. This
pushes each embedding to represent different information from the
others.

This loss can be added to any \gls{scFM}s at pre-training or fine-tuning.

\subsubsection{Hierarchical classifier
loss}\label{hierarchical-classifier-loss}

The goal of the hierarchical classifier is to enable efficient label
predictions for a set of related labels defined by a known graph.

The scPRINT-1 classifier generates predictions for all possible labels
in a hierarchical ontology, while producing logits only for the most
fine-grained elements. To predict the other elements, it only has to
aggregate their children' s logits. We improve this loss
in scPRINT-2 by using the entire ontological graph: e.g., if a cell is
an \emph{olfactory neuron}, then it is also a neuron. If the classifier
predicts \emph{glutaminergic neuron}, it is wrong at this level but
correct for \emph{neuron}, meaning we penalize it less overall than a
non-neuron label, like \emph{fibroblast}. In conjunction
with our weighted sampler, this allows the model to learn rich gradients
from a low volume of data. We also implement two
additional classes for predictions in our hierarchical classifier
compared to scPRINT-1: age and tissue of origin.

During pre-training, we perform label prediction for different classes,
e.g., cell type, disease, assay, age, tissue, ethnicity, sex, and
organism. We created a specific relabeling of the age label that could
be very fine-grained, e.g., 2 weeks, 3 weeks, 35 years old, 36 years
old, into biologically relevant groups such as \emph{embryo,}
\emph{fetal}, \emph{6-month-old}, \emph{1-year-old}, \emph{adolescent},
young adult, and so on. We mapped both human and mouse data this way to
a common age profile. These were the only two species with such labels
available. The labels follow a hierarchy defined by ontologies: the Cell
Ontology for cell type, MONDO for disease, EFO for assay, HANCESTRO for
ethnicity, HSAPDV for age, UBERON for tissue, NCBITaxon for organism,
and EFO for sex\textsuperscript{89--92}. We do not compute the loss for
cells with the unknown label.

The algorithm thus becomes:

Algorithm: L\_class (Hierarchical Classification Loss) (3)

Input:

- pred: predicted logits {[}batch\_size × n\_leaf\_labels{]}

- cl: ground truth labels {[}batch\_size{]}

- labels\_hierarchy: binary matrix {[}n\_parent\_labels ×
n\_leaf\_labels{]}

Output: hierarchical binary cross-entropy loss

1. Initialize target matrix newcl ← zeros{[}batch\_size ×
n\_leaf\_labels{]}

2. Initialize weight matrix weight ← ones{[}batch\_size ×
n\_leaf\_labels{]}

3. // Handle leaf labels (known exact labels)

For each valid leaf label cl{[}i{]} where cl{[}i{]} ∈ {[}0,
n\_leaf\_labels):

newcl{[}i, cl{[}i{]}{]} ← 1

4. // Handle unknown labels

For each unknown label cl{[}i{]} where cl{[}i{]} = -1:

weight{[}i, :{]} ← 0

5. // Handle parent labels (partial knowledge)

If any cl{[}i{]} ≥ n\_leaf\_labels:

parent\_idx ← cl{[}i{]} - n\_leaf\_labels

// Zero out weights for unknown leaf children

weight{[}i, children\_of\_parent{[}parent\_idx{]}{]} ← 0

// Set targets for all possible children to 1

newcl{[}i, children\_of\_parent{[}parent\_idx{]}{]} ← 1

// Compute parent-level predictions and targets

For each parent p:

// Aggregate leaf predictions using logsumexp

addpred{[}p{]} ← logsumexp(pred{[}:, children\_of\_parent{[}p{]}{]})

// Set parent target based on leaf targets

addnewcl{[}p{]} ← max(newcl{[}:, children\_of\_parent{[}p{]}{]})

// Weight inversely proportional to the number of children

addweight{[}p{]} ← addnewcl{[}p{]} /
√\textbar children\_of\_parent{[}p{]}\textbar{}

// Concatenate parent predictions/targets with leaf ones

pred ← concat(pred, addpred)

newcl ← concat(newcl, addnewcl)

weight ← concat(weight, addweight)

6. Return binary\_cross\_entropy\_with\_logits(pred, newcl, weight)

The hierarchical loss is available as a standalone function on GitHub
Gist\footnote{\url{https://gist.github.com/jkobject/5b36bc4807edb440b86644952a49781e}}.

This loss replaces a classical pytorch classifier loss, such as
binary\_cross\_entropy\_with\_logits.

\subsubsection{Variable context length}\label{variable-context-length}

The goal of the variable context length method is to decrease the
model's bias toward a specific number of elements in context.

Indeed, we noticed that at inference time, the model's performance could
be lower in variable-context situations (e.g., on gene-panel datasets or
when using only expressed genes). We thus introduced a
\textbf{variable-context} training scheme in which the model's context
sometimes drops by a random amount (see Table \ref{tab:additive_benchmark}; see \hyperref[methods]{Methods}).
This makes the model less biased toward a specific input context during
inference and decreases training time (see Supplementary Materials). Again, here
we see strong consistent improvement in the model's performance across
our additive benchmark.

This can be applied to any transformer models where the number of
elements in context can be chosen arbitrarily.

\subsubsection{Adversarial classifier
loss}\label{adversarial-classifier-loss}

The goal of the Adversarial classifier is to remove batch
effect\textsuperscript{93,94}.

The adversarial classifier is applied only to the \emph{cell\_type} cell
embedding and is tasked to classify the organism of origin for each
cell. It uses the same \gls{MLP} as regular classifiers (2 layers, 256 as
inner dimension). We use the reverse\_gradient operation on top of a
simple softmax-based binary cross-entropy classifier loss as follows:

Algorithm: L\_adv (adversarial classifier) (4)

Input:

- e: input cell embedding tensor {[}batch\_size × feature\_dim{]}

- c: input ground truth label {[}batch\_size{]}

Output: cross-entropy loss

// reverse the gradient for adversarial behavior

e = grad\_reverse(e)

// compute logits from the embedding using an MLP

logits = MLP(e)

Return cross\_entropy(logits, c)

with

Algorithm: grad\_reverse (5)

Input:

- e: input tensor {[}batch\_size × feature\_dim{]}

- λ: scaling factor for gradient reversal = 1

Output: tensor with reversed gradients

If forward Pass:

1. Return e unchanged (identity function)

If backward Pass:

1. Reverse and scale the incoming gradients

e.grad\_input ← -λ × e.grad\_output

2. Return reversed gradients

Return grad\_input, None // None for λ parameter

We use it to predict both organisms and sequencers. Sequencers are
mapped to a set of coarser labels, as we cannot use the hierarchical
classifier in an adversarial context. Indeed, as it is sigmoid-based, it
could easily set all label logits to -inf.

This loss can be added during pre-training or finetuning of a foundation
model, provided batch labels are available.

\subsection{Additive Benchmark's
datasets}\label{additive-benchmarks-datasets}

The gene network analysis is performed on a test kidney single-cell
dataset, using 1000 cells from the same cell type, and is compared with
the omnipath ground truth (also known as the omnipath benchmark) across
all cell types. It is also performed for 1000 K562 cells, comparing it
to a network assembled from all genes ``i'' whose expression changes
significantly when gene ``j'' is perturbed, using a genome-wide
perturb-seq dataset called GWPS benchmark\textsuperscript{95}.

Knowing that perturb-seq still often implies cell-type- and
patient-specific off-target effects and cannot detect many direct
effects\textsuperscript{96--99}.

The cell type prediction uses accuracy, and batch correction uses scIB
v2, as in Kalfon et al.\textsuperscript{3}. Both the lung and pancreas
datasets have also been used in Kalfon et al. They are test datasets,
removed from the pre-training corpus, and both come from the initial
scIB paper\textsuperscript{100}.

\subsection{\gls{scPRINT}-2}\label{scprint-2-1}

The model architecture is composed of:

\begin{itemize}
\item
  An \textbf{encoder/tokenizer} that takes multiple inputs, such as raw
  expression data, gene names, and gene locations, and embeds them in a
  high-dimensional space used by the transformer.
\item
  A \textbf{trunk} with a bidirectional multi-head transformer, an
  XPressor bidirectional multi-head transformer, and a set of \gls{VAE}s
  applied to each XPressor output embeddings.
\item
  A \textbf{class decoder} that transforms the output cell embeddings of
  the XPressor into cell-specific label prediction logits over a range
  of classes.
\item
  An \textbf{expression decoder} to transform the output embeddings into
  expression values
\end{itemize}

Of the above-cited additive benchmark elements, \gls{scPRINT}-2 contains:
\textbf{XPressor, all databases, denoising}, \textbf{cluster-based
sampling,} \textbf{elastic cell similarity, ZINB+MSE,} \textbf{VAE
compressor, variable context with larger context, TF masking, GNN
expression encoder, and fine-tuned ESM3} (See Supplementary Figures~\ref{fig-s1-illustration-of-the-full-scprint-2s-architecture-input-and-output}, \ref{fig-s9-illustration-of-the-multiple-perturbations-applied-to-expression-data-in-scprint-2}, \ref{fig-s17-illustration-of-criss-cross-attention},
\ref{fig-s18-illustration-of-the-similarity-and-dissimilarity-based-contrastive-losses-used-in-scprint-2})

We now go into some more details about the model:

\subsubsection{Encoder / Tokenizer}\label{encoder-tokenizer}

In \gls{scPRINT}-2, each gene in a cell is converted to an embedding: It
corresponds to the sum of 3 different elements:

1. An embedding representing the gene itself using ESM3 with a
fine-tuning adaptor layer (see
\hyperref[esm3-fine-tuning-gene-encoder]{Methods})

2. An embedding of the gene location in the genome. This helps the model
understand that genes with similar locations tend to be regulated by
similar regulatory regions\textsuperscript{101}, a relationship
well-known in cellular biology.

We encode the genes' locations using positional encoding. Every gene
within 10,000 bp of the next is considered to be in the same location;
otherwise, we increment the location by 1. We do this for all genes in
the Ensembl database per organism.

3. An embedding of the gene expression in the cell and its neighbor
using our \textbf{GNN} (see
\hyperref[gnn-expression-encoder]{Methods})

Finally, during pre-training, a subset of 3200 genes is used to encode a
cell expression profile. If fewer than 3200 genes are expressed in both
the cell and its neighbors, we pad them with randomly sampled
unexpressed genes (meaning with an expression value of 0). This approach
allows the model to see different patches of the same cell profile
during training.

The full set of embeddings of cell i sent to the transformer is the
matrix \(X_{i}\) where

\begin{equation}
X_{i} = [g_{0} + e_{i,0} + l_{0}, g_{1} + e_{i,1} + l_{1}, \ldots]
\end{equation}

Where \(g_{j}\) is the gene j encoding, \(e_{i,j}\) is the encoding of
the expression of gene j in cell i, \(l_{j}\) is the gene j location
encoding.

Additionally, the Xpressor layers will receive a set of learnt prototype
tokens representing the different class-level cell embeddings.

\subsubsection{Trunk}\label{trunk}

The model ``trunk'' is a bidirectional encoder similar to
BERT\textsuperscript{102} with \emph{n} layers, \emph{h} attention
heads, and a dimension of \emph{d}. It uses the
flashattention2\textsuperscript{103} methodology implemented in Triton
to compute its attention matrix. It uses the pre-normalization
technique\textsuperscript{104}, with a sped-up layer norm implemented in
Triton's tutorial\textsuperscript{105}. It uses stochastic depth with
increasing dropout probability\textsuperscript{106} (see
\hyperref[base-model-and-training]{\underline{Base for details about small and
medium-sized models}}).

It has a 2-layer \gls{MLP} with a 4x width increase in its hidden layer and a
GELU activation function.

Each Layer or block is composed, in order, of a layer-norm,
self-attention, layer-norm, MLP, and layer-norm, cross-attention,
layer-norm, MLP, which are only used during the decoding step. It has an
additional m Xpressor blocks/layers applied to its 10 latent cell
tokens.

The output cell embeddings of the Xpressor layers are then compressed
with \gls{VAE}s with respective latent for the {[}cell\_type, tissue, age,
sex, disease, sequencer, ethnicity, organism, cell culture,
additional{]} classes of: 64, 32, 8, 2, 16, 8, 8, 8, 2, None (no VAE)

\subsubsection{Class Decoders}\label{class-decoders}

The class decoders are MLPs applied to compressed representations of
their respective \gls{VAE}s, with a shape of
\(\lbrack\mu_{c},\ 256,\ n_{c}\rbrack\) with \(n_{c}\) the number of
labels in the class c and \(\mu_{c}\) the dimension of this class for
the VAE.

\subsubsection{Expression Decoder}\label{expression-decoder-2}

We had noticed that scPRINT-1 initially produced embeddings that could
be biased by the cell-depth token. We thus push \gls{scPRINT}-2 to be
depth-invariant by introducing the sequencing depth information only in
the Expression Decoder, ensuring that the output gene-cell tokens
contain little absolute sequencing depth information (see Figure \ref{fig:scprint2_denoising}, see
Supplementary Figure~\ref{fig-s1-illustration-of-the-full-scprint-2s-architecture-input-and-output}). This debiases cell embedding to depth data and also
improves denoising (see Table 1).

The expression decoder thus gets applied to the output gene embeddings
and also receives the log2p1-transformed sequencing depth (also called
total cell expression count) \(c\) andis of the form:

\begin{equation}
\mu,\ \theta,\ \pi\  = \ MLP(e\ ||\ c)
\end{equation}

The \gls{MLP} is a two-layer neural network with dimensions {[}\emph{d+1, d},
3{]}, where \textbar\textbar{} denotes the concatenation operation.

The parameters \(\mu, \theta, \pi\) are the parameters of the \gls{ZINB} and
are used in the ZINB+MSE loss.

\subsection{Pre-training}\label{pre-training}

The three main tasks in the multi-task pre-training of \gls{scPRINT}-2 are
denoising, classification, and bottleneck learning. While the denoising
loss enhances the model' s ability to find meaningful
gene-gene connections, the other two try to make the model and its cell
embedding representation more robust and cell-type-specific. The tasks
are presented below.

\subsubsection{Optimization method}\label{optimization-method-2}

Optimization is performed with fused ADAMW and a weight decay of 0.01.
We observed a complete inability to learn when using the base ADAM
algorithm, which has a similar weight decay schedule. This can be
explained by a known inequivalence issue in ADAM\textsuperscript{107}.

We do not use the stochastic weight averaging\textsuperscript{108}
method during training.

During pre-training, the hyperparameters are set to a dropout of 0.1, a
learning rate (LR) of 1e-4, and the precision is set to 16-mixed with
residuals in fp32. We clip gradients to 10 and train over many
sub-epochs of 20,000 training and 20,000 validation batches, with a
warmup of 2,000 steps. Across epochs, we use a linear LR decrease of 0.6
with a patience of 2, and we stop training after 4 consecutive increases
in validation loss. We initialize weights to a normal distribution
around 1, biases to 0, and biases for the final layer of the Classifiers
to -0.12.

Our batch size is 128, and we use a pre-norm strategy for the
transformer with a linearly increasing stochastic depth dropout rate of
0.02 per layer. We use a noise parameter of 70\%. We split the cells in
the datasets into 98\% for training and 2\% for validation, and reserve
at least 2\% of the split datasets for testing. Our reconstruction loss
is \gls{ZINB}+MSE (see the \hyperref[zinbmse-loss]{\underline{ZINB+MSE section in
Methods}}).

While many pre-training variants can be selected from contrastive
learning, classification, adversarial classification, compression (with
XPressor and VAE), masking, biased masking, and imputation, the choice
may depend on specific biological assumptions.

\gls{scPRINT}-2 is trained with denoising an input cell profile, given its
nearest neighbor's expression.

Given the same information, it also performs label prediction during
pre-training for: cell type, disease, sequencer, age, tissue, ethnicity,
sex, cell culture, and organism. The classification task is performed
jointly with the denoising task, meaning that labels are predicted from
corrupted expression data and from nearest-neighbor expression
information. The hierarchical classifier is applied to the
\gls{VAE}s'{} latent embeddings.

During decoding, it regenerates the expression profile for all input
genes, including those dropped during variable context selection. This
effectively does gene imputation.

The decoder receives only the gene location and ESM3 embedding and
performs cross-attention on cell embeddings. The cell embeddings are the
output of the \gls{VAE}s and Xpressor layers, so the input is:

\begin{equation}
X_{i} = [g_{0} + l_{0}, g_{1} + l_{1}, \ldots]
\end{equation}

And cell-embeddings are:

\begin{equation}
C_{i} = [c_{i0}, c_{i1}, \ldots] = \bigcup_{j}\text{VAE}_{j}(u_{ij}), \quad u_{ij} \in U_{j}
\end{equation}
