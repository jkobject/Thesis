\newglossaryentry{latex}
{
        name=latex,
        description={Is a mark up language specially suited for 
scientific documents},
        type=\acronymtype
}

\newglossaryentry{scRNA-seq}
{
        name={scRNA-seq},
        description={Single-Cell RNA-Sequencing: Method to measure the RNA content of a cell},
        type=\acronymtype
}

\newglossaryentry{RNA}
{
    name={RNA},
    description={RiboNucléique Acide. Polymeric molecule essential in various biological roles in coding, decoding, regulation and expression of genes},
    type=\acronymtype
}
\newglossaryentry{GRN}
{
    name={GRN},
    description={Gene Regulatory Network. A collection of molecular regulators that interact with each other and with other substances in the cell to govern the gene expression levels of mRNA and proteins},
    type=\acronymtype
}
\newglossaryentry{GNN}
{
    name={GNN},
    description={Graph Neural Network. A class of artificial neural networks for processing data that can be represented as graphs},
    type=\acronymtype
}
\newglossaryentry{TF}
{
    name={TF},
    description={Transcription Factor. A protein that controls the rate of transcription of genetic information from DNA to messenger RNA, by binding to a specific DNA sequence},
    type=\acronymtype
}
\newacronym{RBP}{RBP}{RNA Binding Protein}
\newacronym{AIVC}{AIVC}{Artificial Intelligence for a Virtual Cell}

\newglossaryentry{DNA}
{
    name={DNA},
    description={Deoxyribonucleic Acid. Molecule that carries genetic information for the development and functioning of an organism},
    type=\acronymtype
}
\newglossaryentry{UCE}
{
    name={UCE},
    description={Universal Cell Embedding. A foundation model for single-cell biology},
    type=\acronymtype
}
\newglossaryentry{ESM}
{
    name={ESM},
    description={Evolutionary Scale Modeling. Protein language models trained on evolutionary data},
    type=\acronymtype
}
\newacronym{ECM}{ECM}{Extracellular Matrix}
\newacronym{SEM}{SEM}{Structural Equation Modeling}
\newglossaryentry{ENCODE}
{
    name={ENCODE},
    description={Encyclopedia of DNA Elements. A public research project which aims to identify all functional elements in the human genome sequence},
    type=\acronymtype
}
\newacronym{EPR}{EPR}{Early Precision Ratio}
\newglossaryentry{KO}
{
    name={KO},
    description={Knockout. A genetic technique in which one of an organism's genes is made inoperative},
    type=\acronymtype
}
\newacronym{SOTA}{SOTA}{State of the Art}
\newacronym{BPH}{BPH}{Benign Prostatic Hyperplasia}
\newglossaryentry{TME}
{
    name={TME},
    description={Tumor Microenvironment. The environment around a tumor, including the surrounding blood vessels, immune cells, fibroblasts, signaling molecules and the extracellular matrix},
    type=\acronymtype
}
\newacronym{gwps}{gwps}{Genome-wide Perturb-seq}
\newglossaryentry{scGPT}
{
    name={scGPT},
    description={Single-cell Generative Pretrained Transformer. A foundation model for single-cell biology based on the GPT architecture},
    type=\acronymtype
}
\newglossaryentry{AI}
{
    name={AI},
    description={Artificial Intelligence. The simulation of human intelligence processes by machines, especially computer systems},
    type=\acronymtype
}
\newglossaryentry{LLM}
{
    name={LLM},
    description={Large Language Model. A language model notable for its ability to achieve general-purpose language generation and understanding},
    type=\acronymtype
}
\newglossaryentry{PE}
{
    name={PE},
    description={Positional Encoding. A mechanism in Transformers to inject information about the relative or absolute position of the tokens in the sequence},
    type=\acronymtype
}
\newacronym{GELU}{GELU}{Gaussian Error Linear Unit}
\newglossaryentry{BERT}
{
    name={BERT},
    description={Bidirectional Encoder Representations from Transformers. A transformer-based machine learning technique for natural language processing pre-training},
    type=\acronymtype
}
\newacronym{LSE}{LSE}{Log-Sum-Exp}
\newglossaryentry{scVI}
{
    name={scVI},
    description={Single-cell Variational Inference. A probabilistic framework for analyzing single-cell RNA sequencing data},
    type=\acronymtype
}
\newacronym{MVC}{MVC}{Model-View-Controller}
\newacronym{ODE}{ODE}{Ordinary Differential Equation}
\newglossaryentry{ARI}
{
    name={ARI},
    description={Adjusted Rand Index. A measure of the similarity between two data clusterings},
    type=\acronymtype
}
\newglossaryentry{NMI}
{
    name={NMI},
    description={Normalized Mutual Information. A normalization of the Mutual Information (MI) score to scale the results between 0 (no mutual information) and 1 (perfect correlation)},
    type=\acronymtype
}
\newglossaryentry{NGS}
{
    name={NGS},
    description={Next-Generation Sequencing. A high-throughput method used to determine the sequence of nucleotides in DNA or RNA samples},
    type=\acronymtype
}
\newglossaryentry{ASW}
{
    name={ASW},
    description={Average Silhouette Width. A measure of how similar an object is to its own cluster (cohesion) compared to other clusters (separation)},
    type=\acronymtype
}
\newglossaryentry{iLISI}
{
    name={iLISI},
    description={Integration Local Inverse Simpson’s Index. A metric to quantify the degree of mixing of datasets in an integrated embedding},
    type=\acronymtype
}
\newglossaryentry{kBET}
{
    name={kBET},
    description={k-nearest-neighbor Batch Effect Test. A metric to quantify batch effects in single-cell RNA-seq data},
    type=\acronymtype
}
\newglossaryentry{ML}
{
    name={ML},
    description={Machine Learning. A field of inquiry devoted to understanding and building methods that 'learn', that is, methods that leverage data to improve performance on some set of tasks},
    type=\acronymtype
}
\newglossaryentry{NN}
{
    name={NN},
    description={Neural Network. A method in artificial intelligence that teaches computers to process data in a way that is inspired by the human brain},
    type=\acronymtype
}
\newacronym{GN}{GN}{Gene Network}
\newacronym{CGT}{CGT}{Cell Graph Transformer}
\newacronym{ASO}{ASO}{Antisense Oligonucleotide}
\newglossaryentry{UMAP}
{
    name={UMAP},
    description={Uniform Manifold Approximation and Projection. A dimension reduction technique that can be used for visualization similarly to t-SNE, but also for general non-linear dimension reduction},
    type=\acronymtype
}
\newglossaryentry{PCA}
{
    name={PCA},
    description={Principal Component Analysis. A statistical procedure that uses an orthogonal transformation to convert a set of observations of possibly correlated variables into a set of values of linearly uncorrelated variables called principal components},
    type=\acronymtype
}
\newglossaryentry{FM}
{
    name={FM},
    description={Foundation Model. A large machine learning model trained on a vast amount of data at scale that can be adapted to a wide range of downstream tasks},
    type=\acronymtype
}
\newglossaryentry{scFM}
{
    name={scFM},
    description={Single-cell Foundation Model. A foundation model specifically designed for single-cell biology tasks},
    type=\acronymtype
}
\newacronym{W2}{W2}{Wasserstein-2 Distance}
\newglossaryentry{PEFT}
{
    name={PEFT},
    description={Parameter-Efficient Fine-Tuning. Approaches to fine-tune large models with a small number of parameters},
    type=\acronymtype
}
\newglossaryentry{VAE}
{
    name={VAE},
    description={Variational Autoencoder. A type of artificial neural network used to learn efficient data codings in an unsupervised manner},
    type=\acronymtype
}
\newacronym{FSQ-VAE}{FSQ-VAE}{Finite Scalar Quantization Variational Autoencoder}
\newglossaryentry{LR}
{
    name={LR},
    description={Learning Rate. A hyperparameter that controls how much to change the model in response to the estimated error each time the model weights are updated},
    type=\acronymtype
}
\newacronym{NNZ}{NNZ}{Number of Non-Zeros}
\newglossaryentry{MSE}
{
    name={MSE},
    description={Mean Squared Error. A measure of the average of the squares of the errors—that is, the average squared difference between the estimated values and the actual value},
    type=\acronymtype
}
\newglossaryentry{ZINB}
{
    name={ZINB},
    description={Zero-Inflated Negative Binomial. A distribution used to model count data that has an excess of zero counts},
    type=\acronymtype
}
\newglossaryentry{KNN}
{
    name={KNN},
    description={K-Nearest Neighbors. A non-parametric method used for classification and regression},
    type=\acronymtype
}
\newacronym{CxG}{CxG}{CellxGene}
\newglossaryentry{CCE}
{
    name={CCE},
    description={Categorical Cross Entropy. A loss function used in multi-class classification tasks},
    type=\acronymtype
}
\newacronym{OR}{OR}{Odds Ratio}
\newglossaryentry{AUPRC}
{
    name={AUPRC},
    description={Area Under the Precision-Recall Curve. A performance metric for binary classification problems},
    type=\acronymtype
}
\newacronym{log1p}{log1p}{Logarithm of (1 + x)}
\newacronym{FFPE}{FFPE}{Formalin-Fixed Paraffin-Embedded}
\newglossaryentry{MMD}
{
    name={MMD},
    description={Maximum Mean Discrepancy. A kernel-based statistical test used to determine whether two given samples are drawn from the same distribution},
    type=\acronymtype
}
\newglossaryentry{MLP}
{
    name={MLP},
    description={Multi-Layer Perceptron. A class of feedforward artificial neural network},
    type=\acronymtype
}
\newglossaryentry{scPRINT}
{
    name={scPRINT},
    description={Single-cell PRe-trained INference Transformer. The model developed in this thesis},
    type=\acronymtype
}
\newglossaryentry{GPU}
{
    name={GPU},
    description={Graphics Processing Unit. A specialized electronic circuit designed to manipulate and alter memory to accelerate the creation of images in a frame buffer intended for output to a display device},
    type=\acronymtype
}
\newglossaryentry{mRNA}
{
    name={mRNA},
    description={Messenger RNA. A single-stranded RNA molecule that corresponds to the genetic sequence of a gene, and is read by a ribosome in the process of synthesizing a protein},
    type=\acronymtype
}
\newglossaryentry{tRNA}
{
    name={tRNA},
    description={Transfer RNA. An RNA molecule that helps decode a messenger RNA (mRNA) sequence into a protein},
    type=\acronymtype
}
\newglossaryentry{rRNA}
{
    name={rRNA},
    description={Ribosomal RNA. A type of non-coding RNA which is the primary component of ribosomes, essential to all cells},
    type=\acronymtype
}
\newglossaryentry{siRNA}
{
    name={siRNA},
    description={Small Interfering RNA. A class of double-stranded RNA non-coding RNA molecules, typically 20-24 base pairs in length, similar to miRNA, and operating within the RNA interference (RNAi) pathway},
    type=\acronymtype
}
\newglossaryentry{scGEN}
{
    name={scGEN},
    description={Single-cell Generative. A tool for predicting single-cell perturbation responses},
    type=\acronymtype
}
\newglossaryentry{miRNA}
{
    name={miRNA},
    description={MicroRNA. A small single-stranded non-coding RNA molecule (containing about 22 nucleotides) found in plants, animals and some viruses, that functions in RNA silencing and post-transcriptional regulation of gene expression},
    type=\acronymtype
}
\newglossaryentry{lncRNA}
{
    name={lncRNA},
    description={Long Non-Coding RNA. A type of RNA, defined as being transcripts with lengths exceeding 200 nucleotides that are not translated into protein},
    type=\acronymtype
}

\newglossaryentry{mFM}
{
    name={mFM},
    description={Molecular Foundation Model. A foundation model trained on molecular data},
    type=\acronymtype
}
\newglossaryentry{nFM}
{
    name={nFM},
    description={Nucleotide Foundation Model. A foundation model trained on nucleotide sequences},
    type=\acronymtype
}
\newglossaryentry{cFM}
{
    name={cFM},
    description={Cell Foundation Model. A foundation model trained on cellular data},
    type=\acronymtype
}
\newglossaryentry{tFM}
{
    name={tFM},
    description={Tissue Foundation Model. A foundation model trained on tissue data},
    type=\acronymtype
}
\newglossaryentry{SMILES}
{
    name={SMILES},
    description={Simplified Molecular Input Line Entry System. A specification in the form of a line notation for describing the structure of chemical species using short ASCII strings},
    type=\acronymtype
}
\newglossaryentry{pLM}
{
    name={pLM},
    description={Protein Language Model. A language model trained on protein sequences},
    type=\acronymtype
}
\newglossaryentry{NLP}
{
    name={NLP},
    description={Natural Language Processing. A subfield of linguistics, computer science, and artificial intelligence concerned with the interactions between computers and human language},
    type=\acronymtype
}
\newglossaryentry{QKV}
{
    name={QKV},
    description={Query, Key, Value. The three matrices used in the attention mechanism of transformers},
    type=\acronymtype
}
\newglossaryentry{LoRA}
{
    name={LoRA},
    description={Low-Rank Adaptation. A technique for fine-tuning large language models},
    type=\acronymtype
}
\newglossaryentry{MSA}
{
    name={MSA},
    description={Multiple Sequence Alignment. The alignment of three or more biological sequences (protein or nucleic acid) of similar length},
    type=\acronymtype
}
\newglossaryentry{IB}
{
    name={IB},
    description={Information Bottleneck. A method for extracting relevant information from an input variable},
    type=\acronymtype
}
\newglossaryentry{VQ-VAE}
{
    name={VQ-VAE},
    description={Vector Quantized Variational Autoencoder. A generative model that learns discrete latent representations},
    type=\acronymtype
}
\newglossaryentry{FSQ}
{
    name={FSQ},
    description={Finite Scalar Quantization. A quantization method for latent representations},
    type=\acronymtype
}
\newglossaryentry{scIB}
{
    name={scIB},
    description={Single-cell Integration Benchmark. A benchmark for single-cell RNA-seq integration methods},
    type=\acronymtype
}
\newglossaryentry{GSEA}
{
    name={GSEA},
    description={Gene Set Enrichment Analysis. A computational method that determines whether an a priori defined set of genes shows statistically significant, concordant differences between two biological states},
    type=\acronymtype
}
\newglossaryentry{PPI}
{
    name={PPI},
    description={Protein-Protein Interaction. The physical contact between two or more protein molecules as a result of biochemical events},
    type=\acronymtype
}
\newglossaryentry{CAF}
{
    name={CAF},
    description={Cancer-Associated Fibroblast. A cell type within the tumor microenvironment that promotes tumorigenic features},
    type=\acronymtype
}
\newglossaryentry{RNA-seq}
{
    name={RNA-seq},
    description={RNA Sequencing. A technique used to analyze the transcriptome of gene expression patterns},
    type=\acronymtype
}
\newglossaryentry{BS-seq}
{
    name={BS-seq},
    description={Bisulfite Sequencing. The use of bisulfite treatment of DNA before routine sequencing to determine the pattern of methylation},
    type=\acronymtype
}
\newglossaryentry{ATAC-seq}
{
    name={ATAC-seq},
    description={Assay for Transposase-Accessible Chromatin using sequencing. A technique used in molecular biology to assess genome-wide chromatin accessibility},
    type=\acronymtype
}
\newglossaryentry{ChIP-seq}
{
    name={ChIP-seq},
    description={Chromatin Immunoprecipitation Sequencing. A method used to analyze protein interactions with DNA},
    type=\acronymtype
}
\newglossaryentry{CRISPR}
{
    name={CRISPR},
    description={Clustered Regularly Interspaced Short Palindromic Repeats. A family of DNA sequences found in the genomes of prokaryotic organisms such as bacteria and archaea, used in gene editing},
    type=\acronymtype
}
\newglossaryentry{scATAC-seq}
{
    name={scATAC-seq},
    description={Single-cell ATAC-seq. A method to map chromatin accessibility at the single-cell level},
    type=\acronymtype
}
\newglossaryentry{OT}
{
    name={OT},
    description={Optimal Transport. A mathematical theory that deals with the problem of finding the most efficient way to move objects from one location to another},
    type=\acronymtype
}
\newglossaryentry{SGD}
{
    name={SGD},
    description={Stochastic Gradient Descent. An iterative method for optimizing an objective function with suitable smoothness properties},
    type=\acronymtype
}
\newglossaryentry{GPT}
{
    name={GPT},
    description={Generative Pre-trained Transformer. A type of large language model (LLM) and a prominent framework for generative artificial intelligence},
    type=\acronymtype
}

\glsaddall
\printglossary[type=\acronymtype, title=Liste des abréviations, toctitle=Liste des abréviations]