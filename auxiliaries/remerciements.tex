\chapter*{Remerciements}
\addcontentsline{toc}{chapter}{Remerciements}

Je tiens à remercier chaleureusement mes collègues du laboratoire ML4IG de l'institut Pasteur ainsi que ceux du centre de science des données de l'ENS et tout particulièrement Jules Samaran, Remi Trimbour et Geert Huizing, pour leur accueil et leur aide durant cette thèse.

\bigskip

Je veux aussi bien sûr remercier Laura Cantini et Gabriel Peyré, mes co-encadrants, pour leur acceuil, leur soutien et leur disponibilité durant ces années. Leur expertise, leurs conseils et leur rigueur scientifique m'ont été d'une aide précieuse dans la réalisation de ce travail. Je suis entêté, je ne suis pas toujours clair, mais ils ont toujours su m'orienter et me guider dans la bonne direction.

\bigskip

Je remercie ma compagne, Juliette Hirsch, pour avoir toujours été à mes côtés ainsi que pour ses conseils, ses questions et son écoute constante. Malgré le stress qui n'a pas toujours fait ressortir le meilleur chez moi, elle a su être le soleil dans mes journées et la patience dans mes doutes.

\bigskip

Je remercie Patrice, Johanna, Marie-Astrid et Emmanuel Kalfon, sans qui je n'en serais pas ici aujourd'hui et qui m'ont conduit, par leur confiance, à faire ce travail. Je remercie Genevieve et Henri Spanjersberg ayant respectivement fait de la recherche en biologie et la même école que moi 50 ans auparavant ont su m'apporter l'intéret pour la science et l'ingérnierie. Je remercie Lucien Kalfon pour m'avoir montré qu'on pouvait rêver à de grandes choses. Finalement, Je tiens à remercier tout particulièrement Monique Obineau, pour son soutien indéfectible, sa bienveillance et son amour inconditionnel, je n'en serais pas là sans elle, son écoute et ses mots.

\bigskip

Je remercie Alex Wolf, Sergei Ribakov, Brice Rafestin et beaucoup d'autres pour leur aide dans le développement informatique des méthodes que je présente ici.

\bigskip

Je remercie les membres de l'institut Pasteur, du département de biologie computationnelle, le hub de bio-informatique, du département de Mathématiques appliquées de l'ENS, ainsi que ceux du supercalculateur Jean Zay sans qui ces méthodes n'auraient pu être entraînées.

\bigskip

Je remercie mes amis, Baptiste Tesson, Oscar Simon, Pier Michele Kaubari, Emile D'allens, Louis Cauquelin, Suzanne Lazarus, Anne-lise Aupetit, Quentin Van Straaten, sans qui je savais d'avance que je n'aurais été capable d'accomplir ce travail.

\bigskip

Je remercie aussi les membres de Nucleate, Whitelab Genomics, Blossom, dot Omics, Biographica et d'autres pour leur intérêt et soutien durant cette thèse.